	\section{野生杂题}
	\begin{lemma}
		设$$u=e^{|x|^k},\ \quad v=\ln(|x|),\ \quad w=|x|^k,\ \quad  y=e^{-|x|^k},\ \quad x\in\mathbb{R}^N\\$$求
		\begin{align*}
			&\Delta u,\ \quad\Delta v,\ \quad \Delta w,\ \quad \Delta y,\ \quad\frac{\partial u}{\partial\nu},\ \quad \frac{\partial v}{\partial \nu},\ \quad \frac{\partial w}{\partial \nu},\ \quad\frac{\partial y}{\partial \nu}\\
			&\nu\text{为$B^1$的单位外法向量}.
		\end{align*}
	\end{lemma}	
	\begin{solution}
		$$u_{x_i}=e^{|x|^{k}}k|x|^{k-1}\frac{x_i}{|x|}=e^{|x|^k}k|x|^{k-2}x_i$$
		$$u_{x_ix_i}=e^{|x|^k}[(k^2|x|^{2k-4}x^2_i)+k|x|^{k-2}+kx^2_i(k-2)|x|^{k-4}]$$
		故
		$$\Delta u=e^{|x|^k}[k^2|x|^{2k-2}+(Nk+k(k-2))|x|^{k-2}]=e[2k^2+Nk-2]$$
		$$\frac{\partial u}{\partial \nu}=\nabla u\cdot\nu=e^{|x|^k}k|x|^{k-1}=ek$$
		{\noindent} \rule[0pt]{12cm}{0.05em}
		$$v_{x_i}=\frac{x_i}{|x|^2},\ v_{x_ix_i}=\frac{|x|^2-2x^2_i}{|x|^4},\ $$
		故
		$$\Delta v=\frac{N|x|^2-2|x|^2}{|x|^4}=\frac{N-2}{|x|^2}=N-2$$
		$$\frac{\partial v}{\partial \nu}=\nabla v\cdot \nu=\frac{1}{|x|}=1$$
		{\noindent} \rule[0pt]{12cm}{0.05em}
		$$w_{x_i}=kx_i|x|^{k-2},\ w_{x_ix_i}=k|x|^{k-2}+k(k-2)x^2_i|x|^{k-4}$$
		故
		$$\Delta=Nk|x|^{k-2}+k(k-2)|x|^{k-2}=k(N+k-2)|x|^{k-2}=k(N+k-2)$$
		$$\frac{\partial w}{\partial \nu}=\nabla w\cdot \nu =\frac{k}{|x|}=k$$
		{\noindent} \rule[0pt]{12cm}{0.05em}
		$$\Delta y=e^{-|x|^2}[k^2|x|^{2k-2}-(Nk+k^2-2k)|x|^{k-2}]=e^{-1}(2k-Nk)$$
		$$\frac{\partial y}{\partial \nu}=e^{-|x|^k}(-k|x|^{k-1})=-ke^{-1}$$
	\end{solution}
	\newpage
	\begin{problem}
		设$u\in C^2(\bar{B})$是如下问题
		$$\begin{cases}
			-\Delta u+4Ne^{|x|^2}u=N\cos|x|,\ &x\in B\\
			u(x)=1,\ &x\in\partial B
		\end{cases}$$
		的解,\ 其中$N\geqslant 3,\ B\subset \mathbb{R}^N$单位球.\\
		证明:1:$0<u(x)\leqslant 1.$\\
		2:$\frac{\partial u}{\partial \nu}(x)\geqslant 1,\ x\in\partial B,\ \nu$为$\partial B$的单位外法向量.
	\end{problem}
	\begin{solution}
		$g(x)=|x|^k,\ \Delta g=k(k+N-2)|x|^{k-2},\ \frac{\partial g}{\partial \nu}=1.$\\
		1.定义线性椭圆算子$$\mathcal{L}u=-\Delta u+4Ne^{|x|^2}u,\ x\in B$$
		令$v(x)=1,\ w(x)=\frac{1}{N^\alpha},\ $当$\alpha$足够大时
		有$$\begin{cases}
			\mathcal{L}v=4Ne^{|x|^2}\geqslant 4N\geqslant N\geqslant N\cos|x|=\mathcal{L}u,\ &x\in B\\
			v(x)=1\geqslant 1=u(x)&x\in \partial B
		\end{cases}$$
		$$\begin{cases}
			\mathcal{L}w=\frac{4Ne^{|x|^2}}{N^\alpha}\leqslant \frac{4N}{N^\alpha}\leqslant N\cos 1\leqslant N\cos|x|=\mathcal{L}u,\ &x\in B\\
			0<w(x)=\frac{1}{N^\alpha}\leqslant 1=u(x),\ &x\in \partial B
		\end{cases}$$
		由比较原理有$0<\frac{1}{N^\alpha}\leqslant u(x)\leqslant 1,\ x\in\bar{B}.$\\
		2.令$v(x)=\frac{1}{2}|x|^2+\frac{1}{2},\ $有
		$$\begin{cases}
			\mathcal{L}v=-\frac{N}{2}+4Ne^{|x|^2}(\frac{1}{2}|x|^2+\frac{1}{2})\geqslant 2N-\frac{N}{2}\geqslant N\geqslant N\cos|x|=\mathcal{L}u,\ &x\in B\\
			v(x)=1=u(x),\ &x\in \partial B\\
			\frac{\partial v}{\partial \nu}(x)=1,\ &x\in \partial B
		\end{cases}$$
		故对任意$x\in \partial B,\ $有
		$$\frac{\partial u}{\partial \nu}(x)=\lim\limits_{\sigma\rightarrow 0^+}\frac{u(x)-u(x-\sigma \nu)}{\sigma}\geqslant\lim\limits_{\sigma\rightarrow 0^+}\frac{v(x)-v(x-\sigma\nu) }{\sigma}=1$$
		因此
		$$\frac{\partial u}{\partial \nu}(x)\geqslant 1.$$
	\end{solution}
	\newpage
	\begin{problem}
		设$f(z)$及$g(z)$在$z=0$解析,\ 且$f(0)\neq 0,\ g(0)=g'(0)=0,\ g''(0)\neq0.$试证:$z=0$为$\frac{f(z)}{g(z)}$的二级极点,\ 且
		$$\underset{z=0}{\text{Res}}\left[\frac{f(z)}{g(z)}\right]=\frac{2f'(0)}{g''(0)}-\frac{2f(0)g'''(0)}{3\left[g''(0)\right]^2}$$
	\end{problem}
	
	\begin{solution}
		由题目所给不难证明$z=0$为$g(z)$的二阶零点,\ 且不是$f(z)$的零点,\ 故为$\frac{f(z)}{g(z)}$的二级极点.
		
		令$g(z)=z^2\varphi(z),\ 0\text{不是}\varphi(z)$的零点且$\varphi(0)\neq0$
		
		$$\underset{z=0}{\text{Res}}\left[\frac{f(z)}{g(z)}\right]=\underset{z=0}{\text{Res}}\left[\frac{f(z)}{z^2\varphi(z)}\right]=\lim\limits_{z\rightarrow 0}\left[z^2\frac{f(z)}{z^2\varphi(z)}\right]'=\lim\limits_{z\rightarrow 0}\frac{f'\varphi-f\varphi'}{\varphi^2}$$
		
		由$g(z)=z^2\varphi(z)$得$\varphi(z)=\frac{g(z)}{z^2}$.
		对该式右端进行两次L'Hosipital可得到$\lim\limits_{z\rightarrow 0}\varphi(0)=\frac{g''(0)}{2}.$
		
		$\lim\limits_{z\rightarrow 0}\varphi'(z)=\lim\limits_{z\rightarrow 0}\frac{zg'-2g}{z^3}$.对该式用三次L'Hospital可得$\lim\limits_{z\rightarrow 0}\varphi'(z)=\frac{g'''(0)}{6}.$
		将上两步求得的极限代入可得最终结果.
	\end{solution}
	\newpage
	\begin{problem}
		已知$\left\{a_n\right\}$满足$a_n>0,\ \lim\limits_{n\rightarrow\infty}\left(a_n^2\sum\limits_{k=1}^{n}a_k\right)=\frac{3}{2}.$求极限$\lim\limits_{n\rightarrow\infty}a_n\sqrt[3]{n}.$
	\end{problem}
	
	\begin{solution}
		设$S_n=\sum\limits_{k=1}^{n}a_k,\ $易知$\left\{S_n\right\}$严格单调递增且$\lim\limits_{n\rightarrow\infty}S_n=+\infty,\ \lim\limits_{n\rightarrow\infty}a_n=0.$不然与题设矛盾.\\
		由Larange中值定理知
		$$S_{n+1}^{\frac{3}{2}}-S_{n}^{\frac{3}{2}}=\frac{3}{2}\left(S_{n+1}-S_n\right)\sqrt{\xi}=\frac{3}{2}a_{n+1}\sqrt{\xi},\ \quad\xi\in\left(S_n,\ S_{n+1}\right)$$
		又
		$$a_{n+1}\sqrt{S_n}<a_{n+1}\sqrt{\xi}<a_{n+1}\sqrt{S_{n+1}}$$
		且
		$$\lim\limits_{n\rightarrow\infty}a_{n+1}\sqrt{S_{n+1}}=\lim\limits_{n\rightarrow\infty}\sqrt{a_n^2\sum\limits_{k=1}^{n}a_k}=\sqrt{\frac{3}{2}}$$
		$$\lim\limits_{n\rightarrow\infty}a_{n+1}\sqrt{S_n}=\sqrt{\lim\limits_{n\rightarrow\infty}\left(a_{n+1}^2S_{n+1}-a_{n+1}^3\right)}=\sqrt{\lim\limits_{n\rightarrow\infty}\left(a_{n+1}^2S_{n+1}\right)}=\sqrt{\frac{3}{2}}$$
		故由夹挤定理知
		$$\lim\limits_{n\rightarrow\infty}a_{n+1}\sqrt{\xi}=\sqrt{\frac{3}{2}}\Rightarrow\lim\limits_{n\rightarrow\infty}S_{n+1}^{\frac{3}{2}}-S_{n}^{\frac{3}{2}}=\frac{3}{2}\sqrt{\frac{3}{2}}$$
		由Stolz定理可知
		\begin{align*}
			\lim\limits_{n\rightarrow\infty}a_n\sqrt[3]{n}&=\lim\limits_{n\rightarrow\infty}\frac{a_n\sqrt{S_n}\sqrt[3]{n}}{\sqrt{S_n}}=\sqrt{\frac{3}{2}}\lim\limits_{n\rightarrow}\frac{\sqrt[3]{n}}{\sqrt{S_n}}\\
			&=\sqrt{\frac{3}{2}}\sqrt[3]{\lim\limits_{n\rightarrow\infty}\frac{n}{S_n^\frac{3}{2}}}=\sqrt{\frac{3}{2}}\sqrt[3]{\lim\limits_{n\rightarrow\infty}\frac{(n+1-n)}{S_{n+1}^{\frac{3}{2}}-S_{n}^{\frac{3}{2}}}}\\
			&=1.
		\end{align*}
	\end{solution}
	\newpage
	\begin{problem}
		计算
		$$\lim\limits_{n\rightarrow\infty}\left(\frac{1}{\sqrt{n^2+1}}+\frac{1}{\sqrt{n^2+2}}+\cdots+\frac{1}{\sqrt{n^2+n}}\right)^n$$
	\end{problem}
	
	\begin{solution}
		$$\text{原式}=e^{\lim\limits_{n\rightarrow\infty}n\ln\left(\frac{1}{\sqrt{n^2+1}}+\frac{1}{\sqrt{n^2+2}}+\cdots+\frac{1}{\sqrt{n^2+n}}\right)}$$
		\begin{align*}
			&\lim\limits_{n\rightarrow\infty}n\ln\sum\limits_{k=1}^{n}\frac{1}{\sqrt{n^2+k}}\\
			=&\lim\limits_{n\rightarrow\infty}n\ln\frac{1}{n}\sum\limits_{k=1}^{n}\frac{1}{\sqrt{1+\frac{k}{n^2}}}\\
			=&\lim\limits_{n\rightarrow\infty}n\ln\frac{1}{n}\sum\limits_{k=1}^{n}\left(1+\frac{k}{n^2}\right)^{-\frac{1}{2}}\\
			=&\lim\limits_{n\rightarrow\infty}n\ln\frac{1}{n}\sum\limits_{k=1}^{n}\left[1-\frac{k}{2n^2}+o\left(\frac{1}{n^2}\right)\right]\\
			=&\lim\limits_{n\rightarrow\infty}n\ln\frac{1}{n}\left[n-\frac{n(n+1)}{4n^2}+o\left(\frac{1}{n}\right)\right]\\
			=&\lim\limits_{n\rightarrow\infty}n\ln\left[1-\frac{n+1}{4n^2}+o\left(\frac{1}{n^2}\right)\right]\\
			=&\lim\limits_{n\rightarrow\infty}n\left(-\frac{n+1}{4n^2}\right)\\
			=&-\frac{1}{4}
		\end{align*}
		故原式结果为$e^{-\frac{1}{4}}.$
	\end{solution}
	\newpage
	\begin{problem}
		已知$a_n=\int_{0}^{1}x(1-x^3)^n\,\text{d}x,\ $求$\lim\limits_{n\rightarrow\infty}\frac{a_{n+1}}{a_n}.$
	\end{problem}
	
	\begin{solution}
		法I:
		\begin{align*}
			a_{n+1}&=\int_{0}^{1}x(1-x^3)^n\,\text{d}x=\frac{1}{2}\int_{0}^{1}(1-x^3)^{n+1}\,\text{d}x^2\\
			&=\left.\frac{1}{2}x^2(1-x^3)^{n+1}\right|_{0}^{1}+\frac{3(n+1)}{2}\int_{0}^{1}x^4(1-x^3)^n\,\text{d}x\\
			&=\frac{3(n+1)}{2}\int_{0}^{1}[x-x(1-x^3)](1-x^3)^n\,\text{d}x\\
			&=\frac{3(n+1)}{2}a_n-\frac{3(n+1)}{2}a_{n+1}\\
			&=\frac{3n+3}{3n+5}a_n
		\end{align*}
		故
		$$\lim\limits_{n\rightarrow\infty}\frac{a_{n+1}}{a_n}=1.$$
		法II:
		\begin{align*}
			a_n&=\int_{0}^{1}x(1-x^3)^n\,\text{d}x\left[\text{换元令}t=x^3\right]\\
			&=\frac{1}{3}\int_{0}^{1}x^{-\frac{1}{3}}(1-x)^n\,\text{d}x\\
			&=\frac{1}{3}\mathrm {B}  \left(\frac{2}{3},\ n+1\right)
		\end{align*}
		\begin{align*}
			\lim\limits_{n\rightarrow\infty}\frac{a_{n+1}}{a_n}&=\lim\limits_{n\rightarrow\infty}\frac{\mathrm {B}  \left(\frac{2}{3},\ n+2\right)}{\mathrm {B}  \left(\frac{2}{3},\ n+1\right)}\\
			&=\lim\limits_{n\rightarrow\infty}\frac{\Gamma\left(n+2\right)\Gamma\left(n+\frac{5}{3}\right)}{\Gamma\left(n+1\right)\Gamma(n+\frac{8}{3})}\\
			&=\lim\limits_{n\rightarrow\infty}\frac{n+1}{n+\frac{5}{3}}=1.
		\end{align*}
	\end{solution}
	\newpage
	\begin{problem}
		设函数$f(x)$在区间$\left[0,\ 1\right]$上连续且单调递减,\ 证明:当$0\le \lambda\le 1$时,\ 
		$$\int_{0}^{\lambda}f(x)\,\text{d}x\ge \lambda\int_{0}^{1}f(x)\,\text{d}x.$$
	\end{problem}
	
	\begin{proof}
		法1:构造辅助函数
		$$F(x)=\lambda\int_{0}^{x}\left[f(\lambda t)-f(t)\right]\,\text{d}t,\ \quad x \in \left[0,\ 1\right],\ $$
		则有
		$$F'(x)=\lambda\left[f(\lambda x)-f(x)\right].$$
		因为$\lambda\in\left[0,\ 1\right],\ $所以$x\ge \lambda x\ge 0,\ $又$f(x)$单调递减,\ 所以$f(\lambda x)\ge f(x).$于是$F'(x)\ge 0,\ $即$F(x)$单调递增,\ 故$F(1)\ge F(0)=0,\ $即$\lambda\int_{0}^{1}\left[f(\lambda x)-f(x)\right]\,\text{d}t\ge 0.$原题得证.
		~\\
		
		法2:因为
		$$\lambda\int_{0}^{1}f(x)\,\text{d}x=\lambda\left[\int_{0}^{\lambda}f(x)\,\text{d}x+\int_{\lambda}^{1}f(x)\,\text{d}x\right]=\lambda\int_{0}^{\lambda}f(x)\,\text{d}x+\lambda\int_{\lambda}^{1}f(x)\,\text{d}x,\ $$
		所以
		$$\int_{0}^{\lambda}f(x)\,\text{d}x-\lambda\int_{0}^{1}f(x)\,\text{d}x=(1-\lambda)\int_{0}^{\lambda}f(x)\,\text{d}x-\lambda\int_{\lambda}^{1}f(x)\,\text{d}x.$$
		由积分中值定理,\ 存在$\xi\in\left[0,\ \lambda\right],\ \eta\in\left[\lambda,\ 1\right],\ $使
		$$\int_{0}^{\lambda}f(x)\,\text{d}x=f(\xi)\lambda,\ \quad\int_{\lambda}^{1}f(x)\,\text{d}x=f(\eta)(1-\lambda).$$
		又因为$f(x)$单调递减,\ $\xi\le\eta,\ $知$f(\xi)\ge f(\eta).$于是
		$$(1-\lambda)\int_{0}^{\lambda}f(x)\,\text{d}x-\lambda\int_{\lambda}^{1}f(x)\,\text{d}x=\lambda(1-\lambda)\left[f(\xi)-f(\eta)\right]\ge 0,\ $$
		即$\int_{0}^{\lambda}f(x)\,\text{d}x\ge\lambda\int_{0}^{1}f(x)\,\text{d}x.$原题得证.
	\end{proof}
	\newpage
	\begin{problem}数列$\left\{a_n\right\} = \sqrt{2},\ a_{n+1}=\sqrt{2+a_n},\ $求极限$$\lim\limits_{n\rightarrow \infty}\frac{2^n}{a_1a_2\dots a_n}$$
	\end{problem}
	
	\begin{solution}
		令$a_n = 2\cos b_n$首先$b_1 = \frac{\pi}{4}$
		
		$$2\cos b_{n+1} = \sqrt{2+2\cos b_n}=2\cos \frac{b_n}{2}\Rightarrow a_n = 2\cos \frac{\pi}{2^{n+1}}$$
		
		$$\sin \pi = 2\sin \frac{\pi}{2}\cos \frac{\pi}{2} = 2^2\sin\frac{\pi}{2^2}\cos\frac{\pi}{2^2}\cos \frac{\pi}{2}=\dots=2^n\sin\frac{\pi}{2^n}\prod\limits_{k=1}^{n}\cos \frac{\pi}{2^k}$$
		
		$$\Rightarrow\lim\limits_{n\rightarrow\infty}\prod\limits_{k=1}^{n}\cos \frac{\pi}{2^{k+1}}=\lim\limits_{x\rightarrow \pi}\frac{\sin x}{x\cos\frac{x}{2}} = \frac{2}{\pi}$$ 
		
		这里仅对最后一步化简进行详细展开,\ 将$\prod\limits_{k=1}^{n}\cos \frac{\pi}{2^{k+1}}$表示成$\sin \pi$的形式有
		
		$$\prod\limits_{k=1}^{n}\cos \frac{\pi}{2^{k+1}} = \frac{\sin \pi \cos \frac{\pi}{2^{n+1}}}{2^n\sin \frac{\pi}{2^n}\cos\frac{\pi}{2}}$$
		
		当$n\rightarrow \infty$时,\ $\cos \frac{\pi}{2^{n+1}}\rightarrow 1$,\ $\sin \frac{\pi}{2^n}\sim \frac{\pi}{2^n}.$
	\end{solution}
	\newpage
	\begin{problem}
		证明下述极限:$$\lim\limits_{n\rightarrow\infty}n\left[\left(\frac{b-a}{n}\sum\limits_{k=1}^{n}f\left(a+\frac{b-a}{n}k\right)-\int_{a}^{b}f(x)dx\right)\right]=\frac{b-a}{2}\left[f(b)-f(a)\right]$$
	\end{problem}
	
	\textbf{证明:}令
	$$x_k=a+\frac{b-a}{n}k,\ x_{k-1}=a+\frac{b-a}{n}(k-1),\ x_k-x_{k-1}=\frac{b-a}{n}$$
	因此
	\begin{align*}  
		&\lim\limits_{n\rightarrow\infty}n\left[\left(\frac{b-a}{n}\sum\limits_{k=1}^{n}f\left(a+\frac{b-a}{n}k\right)-\int_{a}^{b}f(x)\text{d}x\right)\right]\\
		&=\lim\limits_{n\rightarrow\infty}n\left[\left(x_k-x_{k-1}\sum\limits_{k=1}^{n}f\left(x_k\right)-\int_{a}^{b}f(x)\text{d}x\right)\right]\\
		&=\lim\limits_{n\rightarrow\infty}n\left[\sum\limits_{k=1}^{n}\int_{x_{k-1
		}}^{x_k}f(x_k)\text{d}x-\sum\limits_{k=1}^{n}\int_{x_{k-1
		}}^{x_k}f(x)\text{d}x\right]\\
		&=\lim\limits_{n\rightarrow\infty}n\sum\limits_{k=1}^{n}\int_{x_{k-1
		}}^{x_k}(f(x_k)-f(x))\text{d}x\\
		&=\lim\limits_{n\rightarrow\infty}n\sum\limits_{k=1}^{n}\int_{x_{k-1
		}}^{x_k}\left[f'(\xi_1)(x_k-x)\text{d}x\right]\left[\text{Larange中值定理}\right]\\
		&=\lim\limits_{n\rightarrow\infty}n\sum\limits_{k=1}^{n}f'(\xi_2)\int_{x_{k-1
		}}^{x_k}\left[(x_k-x)\text{d}x\right]\left[\text{积分第一中值定理}\right]\\
		&=\lim\limits_{n\rightarrow\infty}n\sum\limits_{k=1}^{n}\left[f'(\xi_2)\frac{(x_k-x_{k-1})^2}{2}\right]\\
		&=\lim\limits_{n\rightarrow\infty}n\sum\limits_{k=1}^{n}\left[f'(\xi_2)\frac{(b-a)^2}{2n^2}\right]\\
		&=\lim\limits_{n\rightarrow\infty}\frac{b-a}{2}\frac{b-a}{n}\sum\limits_{k=1}^{n}f'(\xi_2)\\
		&=\frac{b-a}{2}\int_{a}^{b}f'(x)\text{d}x\\
		&=\frac{b-a}{2}\left[f(b)-f(a)\right]
	\end{align*}
	\newpage
	\begin{problem}
		求极限$$\lim\limits_{n\rightarrow \infty}\left(\frac{1}{2n\sqrt{n-1}}+\frac{\sqrt{2}}{2n\sqrt{n-\frac{1}{2}}}+\dots+\frac{\sqrt{n}}{2n\sqrt{n-\frac{1}{n}}}\right)$$
	\end{problem}
	
	\begin{solution}
		这里只对这个数列的一般项进行处理,\ 变化成定积分定义的形式:
		
		$$\text{当}n\rightarrow\infty\text{时},\ \frac{\sqrt{k}}{2n\sqrt{n-\frac{1}{k}}}\sim\frac{\sqrt{k}}{2n\sqrt{n}}$$
		
		故$$\text{原式}=\lim\limits_{n\rightarrow\infty}\frac{1}{2}\sum\limits_{k=1}^{n}\frac{1}{n}\frac{\sqrt{k}}{\sqrt{n}}=\frac{1}{2}\int_{0}^{1}\sqrt{x}\,\mathrm{d}x=\frac{1}{3}$$
	\end{solution}
	\newpage
	\begin{problem}
		证明级数
		$$\sum\limits_{n=1}^{\infty}\frac{1+\frac{1}{2}+\cdots+\frac{1}{n}}{n(n+2)}$$
		收敛并求其和.
	\end{problem}
	
	\begin{solution}
		令
		$$s_n=1+\frac{1}{2}+\cdots+\frac{1}{n}.$$
		由Stolz定理可知
		$$
		\lim\limits_{n\rightarrow\infty}\frac{s_nn^\frac{3}{2}}{n(n+1)}=\lim\limits_{n\rightarrow\infty}\frac{s_nn^\frac{3}{2}}{n^2}=\lim\limits_{n\rightarrow\infty}\frac{s_{n+1}-s_n}{(n+1)^{\frac{1}{2}}-n^{\frac{1}{2}}}=\lim\limits_{n\rightarrow\infty}\frac{\sqrt{n+1}+\sqrt{n}}{n+1}=0.
		$$
		由$\mathrm{P}$-级数判别法知该级数收敛.\\
		~\\
		令$$a_n=\frac{s_n}{n(n+2)}.$$
		\begin{align*}
			\sum\limits_{k=1}^{\infty}a_n&=\frac{1}{2}\left[s_n\left(\frac{1}{n}-\frac{1}{n+2}\right)\right]\\
			&=\frac{1}{2}\left[s_1\left(1-\frac{1}{3}\right)+s_2\left(\frac{1}{2}-\frac{1}{4}\right)+s_3\left(\frac{1}{3}-\frac{1}{5}\right)+s_4\left(\frac{1}{4}-\frac{1}{6}\right)+\cdots\right]\\
			&=\frac{1}{2}\left[s_1+\frac{1}{2}s_2+\frac{1}{3}(s_3-s_1)+\frac{1}{4}(s_4-s_2)
			+\cdots\right]\\
			&=\frac{1}{2}\left[1+\frac{3}{4}+\sum\limits_{k=3}^{\infty}\left(\frac{1}{k-1}-\frac{1}{k}\right)\frac{1}{k}\right]\\
			&=\frac{1}{2}\left[\frac{7}{4}+\sum\limits_{k=3}^{\infty}\left(\frac{1}{k-1}-\frac{1}{k}\right)+\sum\limits_{k=3}^{\infty}\frac{1}{k^2}\right]\\
			&=\frac{1}{2}+\frac{\pi^2}{12}
		\end{align*}
	\end{solution}
	
	\newpage
	\begin{problem}
		计算积分:
		$$\int_{0}^{\frac{\pi}{2}}\frac{t}{\sin t}\,\text{d}t.$$
	\end{problem}
	
	\begin{solution}
		令$t=2\arctan x,\ $则
		$$\text{d}t=\frac{2}{1+x^2}\,\text{d}x,\ \quad x=\tan\frac{t}{2}$$
		以及
		$$\sin t=2\sin\frac{t}{2}\cos\frac{t}{2}=2\frac{\tan\frac{t}{2}}{\sec^2\frac{t}{2}}=\frac{2x}{1+x^2}$$
		于是
		$$\text{原式}=\int_{0}^{1}\frac{2\arctan x}{\frac{2x}{1+x^2}}\cdot\frac{2}{1+x^2}\,\text{d}x=2\int_{0}^{1}\frac{\arctan x}{x}\,\text{d}x$$
		其中$\arctan x$用级数展开:
		$$\arctan x=x-\frac{x^3}{3}+\frac{x^5}{5}\cdots+(-1)^n\frac{x^{2n-1}}{2n+1}+\cdots=\sum\limits_{n=0}^{\infty}\frac{(-1)^n}{2n+1}x^{2n-1}\quad (x^2<1)$$
		把$\arctan x$的展开式代入,\ 并逐项积分,\ 得到
		$$\int_{0}^{1}\frac{\arctan x}{x}\,\text{d}x=\int_{0}^{1}\sum\limits_{n=0}^{\infty}\frac{(-1)^n}{2n+1}x^{2n}\,\text{d}x$$
		根据积分号与求和号可以互换的原则,\ 得到
		$$\int_{0}^{1}\frac{\arctan x}{x}\,\text{d}x=\sum\limits_{n=0}^{\infty}\frac{(-1)^n}{(2n+1)^2}=G$$
		因此得到
		$$\int_{0}^{\frac{\pi}{2}}\frac{t}{\sin t}\,\text{d}t=2G$$
		此处$G$为卡塔兰常数.
	\end{solution}
	\newpage
	\begin{problem}
		设$$I_n=\int \frac{\,\text{d}x}{x^n\sqrt{1+x^2}}$$试建立递推公式.
	\end{problem}
	
	\begin{solution}
		$$\begin{aligned}
			I_n&=\int \frac{x}{x^{n+1}\sqrt{1+x^2}}\,\text{d}x=\int\frac{1}{x^{n+1}}\,\text{d}\sqrt{x^2+1}\\
			&=\frac{\sqrt{x^2+1}}{x^{n+1}}+\int\sqrt{x^2+1}\frac{n+1}{x^{n+2}}\,\text{d}x\\
			&=\frac{\sqrt{x^2+1}}{x^{n+1}}+(n+1)\int\frac{x^2+1}{\sqrt{x^2+1}x^{n+2}}\,\text{d}x\\
			&=\frac{\sqrt{x^2+1}}{x^{n+1}}+(n+1)\left[\int\frac{1}{x^{n+2}\sqrt{x^2+1}}\,\text{d}x+\int\frac{1}{x^n\sqrt{x^2+1}}\,\text{d}x\right]\\
			&=\frac{\sqrt{x^2+1}}{x^{n+1}}+(n+1)\left(I_n+2+I_n\right)\\
		\end{aligned}
		$$
		
		故$nI_n+(n+1)I_{n+2}=-\frac{\sqrt{x^2+1}}{x^{n+1}}.$
	\end{solution}
	
	\newpage
	\begin{problem}
		当  $\|\boldsymbol{H}\|_{\infty}<1  $时,\   $\left\|(\boldsymbol{I}-\boldsymbol{L})^{-1} \boldsymbol{U}\right\|_{\infty} \leqslant\|\boldsymbol{H}\|_{\infty}<1 ,\ $ 因而$\mathrm{Seidel}$迭代法收敛.
	\end{problem}
	
	\begin{solution}
		\begin{equation}
			x^{(k+1)}=(I-L)^{-1}Ux^{(k)}+(I-L)^{-1}g\label{eq1}
		\end{equation}
		利用误差向量的定义知
		$$y_{i}^{(k+1)}=\sum_{j=1}^{i-1} h_{i j} y_{j}^{(k+1)}+\sum_{j=i}^{n} h_{i j} y_{j}^{(k)} .$$
		于是
		$$\left|y_{i}^{(k+1)}\right| \leqslant \sum_{j=1}^{i-1}\left|h_{i j}\right|\left|y_{j}^{(k+1)}\right|+\sum_{j=i}^{n}\left|h_{i j}\right|\left|y_{j}^{(k)}\right|$$
		令
		$$\beta_{i}=\sum_{j=1}^{i-1}\left|h_{i j}\right|,\  \quad \gamma_{i}=\sum_{j=i}^{n}\left|h_{i j}\right|,\  \quad\left|y_{i_{0}}^{(k+1)}\right|=\max _{i}\left|y_{i}^{(k+1)}\right|$$
		由此得到
		$$\left\|\boldsymbol{y}^{(k+1)}\right\|_{\infty} \leqslant \frac{\gamma_{i_{0}}}{1-\beta_{i_{0}}}\left\|\boldsymbol{y}^{(k)}\right\|_{\infty} \leqslant \max _{i} \frac{\gamma_{i}}{1-\beta_{i}}\left\|\boldsymbol{y}^{(k)}\right\|_{\infty}=\mu^{\prime}\left\|\boldsymbol{y}^{(k)}\right\|_{\infty},\ $$
		此处,\   $\mu^{\prime}=\max _{i} \frac{\gamma_{i}}{1-\beta_{i}} .$ 由上式和\eqref{eq1}式可以得到 $ \left\|(\boldsymbol{I}-\boldsymbol{L})^{-1} \boldsymbol{U}\right\|_{\infty} \leqslant \mu^{\prime} .$ 现在证明  $\mu^{\prime} \leqslant \mu .$ 事实上,\  由  $\beta_{i}+\gamma_{i} \leqslant\|\boldsymbol{H}\|_{\infty}<1 $ 得
		$$\beta_{i}+\gamma_{i}-\frac{\gamma_{i}}{1-\beta_{i}}=\frac{\beta_{i}\left(1-\beta_{i}-\gamma_{i}\right)}{1-\beta_{i}} \geqslant 0,\ $$
		于是
		$$\beta_{i}+\gamma_{i} \geqslant \frac{\gamma_{i}}{1-\beta_{i}} .$$
		上式两端对$  1 \leqslant i \leqslant n  $取最大值,\  便得到  $\mu^{\prime} \leqslant \mu=\|\boldsymbol{H}\|_{\infty} .$
	\end{solution}
	\newpage
	\begin{problem}
		若  $\|\boldsymbol{H}\|_{1}<1 ,\ $ 则$\mathrm{Seidel}$迭代法收敛.
	\end{problem}
	
	\begin{solution}
		记  $\overline{\boldsymbol{H}}=(\boldsymbol{I}-\boldsymbol{L})^{-1} \boldsymbol{U} .$ 设  $\lambda$  为  $\overline{\boldsymbol{H}}$  的特征值,\   $\boldsymbol{x}^{\mathrm{T}}$  为其相应的左特征向 量,\  即
		$$\boldsymbol{x}^{\mathrm{T}} \overline{\boldsymbol{H}}=\lambda \boldsymbol{x}^{\mathrm{T}}$$
		或者
		$$\boldsymbol{x}^{\mathrm{T}}(\boldsymbol{I}-\boldsymbol{L})^{-1} \boldsymbol{U}=\lambda \boldsymbol{x}^{\mathrm{T}} .$$
		令  $\boldsymbol{y}^{\mathrm{T}}=\boldsymbol{x}^{\mathrm{T}}(\boldsymbol{I}-\boldsymbol{L})^{-1} ,\ $ 则  $\boldsymbol{x}^{\mathrm{T}}=\boldsymbol{y}^{\mathrm{T}}(\boldsymbol{I}-\boldsymbol{L}) ,\ $ 于是有
		\begin{equation}
			\boldsymbol{y}^{\mathrm{T}} \boldsymbol{U}=\lambda \boldsymbol{y}^{\mathrm{T}}(\boldsymbol{I}-\boldsymbol{L})\label{eq2}
		\end{equation}
		设  $\left|y_{j_{0}}\right|=\max _{1 \leqslant j \leqslant n}\left|y_{j}\right| .$ 考虑 \eqref{eq2}的第  $j_{0}$  个分量有
		$$\sum_{i=1}^{j_{0}} h_{i j_{0}} y_{i}=\lambda\left(y_{j_{0}}-\sum_{i=j_{0}+1}^{n} h_{i j_{0}} y_{i}\right)$$
		由此得
		$$\begin{aligned}
			\left|y_{j_{0}}\right| \sum_{i=1}^{j_{0}}\left|h_{i j_{0}}\right| & \geqslant\left|\sum_{i=1}^{j_{0}} h_{i j_{0}} y_{i}\right|=|\lambda|\left|y_{j_{0}}-\sum_{i=j_{0}+1}^{n} h_{i j_{0}} y_{i}\right| \\
			& \geqslant|\lambda|\left(\left|y_{j_{0}}\right|-\sum_{i=j_{0}+1}^{n}\left|h_{i j_{0}}\right|\left|y_{i}\right|\right) \\
			& \geqslant|\lambda|\left|y_{j_{0}}\right|\left(1-\sum_{i=j_{0}+1}^{n}\left|h_{i j_{0}}\right|\right),\ 
		\end{aligned}$$
		于是
		$$|\lambda| \leqslant \frac{\sum_{i=1}^{j_{0}}\left|h_{i j_{0}}\right|}{1-\sum_{i=j_{0}+1}^{n}\left|h_{i j_{0}}\right|}$$
		注意到
		$$\sum_{i=1}^{j_{0}}\left|h_{i j_{0}}\right|+\sum_{i=j_{0}+1}^{n}\left|h_{i j_{0}}\right|=\sum_{i=1}^{n}\left|h_{i j_{0}}\right| \leqslant\|H\|_{1}<1,\ $$
		有
		$$|\lambda| \leqslant \frac{\sum_{i=1}^{j_{0}}\left|h_{i j_{0}}\right|}{1-\sum_{i=j_{0}+1}^{n}\left|h_{i j_{0}}\right|}<\frac{\sum_{i=1}^{j_{0}}\left|h_{i j_{0}}\right|}{\sum_{i=1}^{n}\left|h_{i j_{0}}\right|-\sum_{i=j_{0}+1}^{n}\left|h_{i j_{0}}\right|}=\frac{\sum_{i=1}^{j_{0}}\left|h_{i j_{0}}\right|}{\sum_{i=1}^{j_{0}}\left|h_{i j_{0}}\right|}=1 .$$
		从而可知
		$$\rho(\overline{\boldsymbol{H}})<1,\ $$
		即$\mathrm{Seidel}$迭代法收敛. 
	\end{solution}
	\newpage
	\begin{problem}
		计算$$I=\lim\limits_{x \rightarrow 0} \frac{\left(\mathrm{e}^{\sin x}+\tan x\right)^{\frac{1}{x}}-\left(\mathrm{e}^{\tan x}+\sin x\right)^{\frac{1}{x}}}{x^{3}} .$$
	\end{problem}
	\begin{solution}
		分子取对数有  $\left(\mathrm{e}^{\sin x}+\tan x\right)^{\frac{1}{x}}-\left(\mathrm{e}^{\tan x}+\sin x\right)^{\frac{1}{x}}=\mathrm{e}^{\frac{\ln \left(\mathrm{e}^{\sin x}+\tan x\right)}{x}}-\mathrm{e}^{\frac{\ln \left(\mathrm{e}^{\tan x}+\sin x\right)}{x}} .$ 由拉格朗日中值定理得
		$$\mathrm{e}^{\frac{\ln \left(\mathrm{e}^{\sin x}+\tan x\right)}{x}}-\mathrm{e}^{\frac{\ln \left(\mathrm{e}^{\tan x}+\sin x\right)}{x}}=f^{\prime}(\xi)\left[\frac{\ln \left(\mathrm{e}^{\sin x}+\tan x\right)}{x}-\frac{\ln \left(\mathrm{e}^{\tan x}+\sin x\right)}{x}\right]$$
		其中  $f(x)=\mathrm{e}^{x},\  f^{\prime}(x)=\mathrm{e}^{x},\ $ $\xi $ 介于  $\frac{\ln \left(\mathrm{e}^{\sin x}+\tan x\right)}{x}$  与  $\frac{\ln \left(\mathrm{e}^{\tan x}+\sin x\right)}{x} $ 之间. 因为
		$$\begin{array}{l}
			\lim\limits_{x \rightarrow 0} \frac{\ln \left(\mathrm{e}^{\sin x}+\tan x\right)}{x}=\lim\limits_{x \rightarrow 0} \frac{\mathrm{e}^{\sin x}+\tan x-1}{x}=\lim\limits _{x \rightarrow 0} \frac{\mathrm{e}^{\sin x}-1}{x}+\lim\limits _{x \rightarrow 0} \frac{\tan x}{x}=2 \\
			\lim\limits_{x \rightarrow 0} \frac{\ln \left(\mathrm{e}^{\tan x}+\sin x\right)}{x}=\lim\limits _{x \rightarrow 0} \frac{\mathrm{e}^{\tan x}+\sin x-1}{x}=\lim\limits _{x \rightarrow 0} \frac{\mathrm{e}^{\tan x}-1}{x}+\lim\limits _{x \rightarrow 0} \frac{\sin x}{x}=2
		\end{array}$$
		所以 $ \lim\limits_{x \rightarrow 0} \xi=2 ,\ $ 则  $I=\mathrm{e}^{2} \lim\limits_{x \rightarrow 0} \frac{\ln \left(\mathrm{e}^{\sin x}+\tan x\right)-\ln \left(\mathrm{e}^{\tan x}+\sin x\right)}{x^{4}} .$
		由拉格朗日中值定理得
		$$\ln \left(\mathrm{e}^{\sin x}+\tan x\right)-\ln \left(\mathrm{e}^{\tan x}+\sin x\right)=g^{\prime}(\eta)\left[\left(\mathrm{e}^{\sin x}+\tan x\right)-\left(\mathrm{e}^{\tan x}+\sin x\right)\right]$$
		其中  $g(x)=\ln x,\  g^{\prime}(x)=\frac{1}{x},\ $ $\eta $ 介于  $\mathrm{e}^{\sin x}+\tan x $ 与 $ \mathrm{e}^{\tan x}+\sin x $ 之间.
		因为
		$$\begin{array}{l}
			\lim _{x \rightarrow 0}\left(\mathrm{e}^{\sin x}+\tan x\right)=1 \\
			\lim _{x \rightarrow 0}\left(\mathrm{e}^{\tan x}+\sin x\right)=1
		\end{array}$$
		所以 $ \lim\limits_{x \rightarrow 0} \eta=1 ,\ $ 则  $I=\mathrm{e}^{2} \lim\limits_{x \rightarrow 0} \frac{\left(\mathrm{e}^{\sin x}-\sin x\right)-\left(\mathrm{e}^{\tan x}-\tan x\right)}{x^{4}} .$
		由拉格朗日中值定理得
		$$\left(\mathrm{e}^{\sin x}-\sin x\right)-\left(\mathrm{e}^{\tan x}-\tan x\right)=h^{\prime}(\zeta)(\sin x-\tan x)$$
		其中  $h(x)=\mathrm{e}^{x}-x,\  h^{\prime}(x)=\mathrm{e}^{x}-1,\ $ $\zeta$  介于  $\sin x $ 与  $\tan x $ 之间.
		因为
		$$\begin{array}{l}
			\sin x \sim x \\
			\tan x \sim x
		\end{array}$$
		所以  $\zeta \sim x ,\ $ 则 $ I=\mathrm{e}^{2} \lim\limits_{x \rightarrow 0} \frac{\left(\mathrm{e}^{\zeta}-1\right)(\sin x-\tan x)}{x^{4}}=\mathrm{e}^{2} \lim\limits_{x \rightarrow 0} \frac{x \cdot\left(-\frac{1}{2} x^{3}\right)}{x^{4}}=-\frac{\mathrm{e}^{2}}{2} .$ 综上所述
		$$I=\lim\limits _{x \rightarrow 0} \frac{\left(\mathrm{e}^{\sin x}+\tan x\right)^{\frac{1}{x}}-\left(\mathrm{e}^{\tan x}+\sin x\right)^{\frac{1}{x}}}{x^{3}}=-\frac{\mathrm{e}^{2}}{2}$$ 
	\end{solution}
	\newpage
	\begin{problem}
		计算积分:
		$$\int_{0}^{\infty}\sin x^n\,\mathrm{d}x$$
	\end{problem}
	\begin{solution}
		法一:
		$$\begin{aligned}
			\int_{0}^{\infty} \sin \left(x^{n}\right) d x & =\frac{1}{n} \int_{0}^{\infty} x^{\frac{1}{n}-1} \sin (x) d x \quad\left(x^{n} \mapsto x\right) \\
			& =\frac{1}{n \Gamma\left(1-\frac{1}{n}\right)} \int_{0}^{\infty}\left(\int_{0}^{\infty} u^{-\frac{1}{n}} e^{-x u} d u\right) \sin (x) d x \\
			& =\frac{1}{n \Gamma\left(1-\frac{1}{n}\right)} \int_{0}^{\infty} u^{-\frac{1}{n}}\left(\int_{0}^{\infty} e^{-x u} \sin (x) d x\right) d u \\
			& =\frac{1}{n \Gamma\left(1-\frac{1}{n}\right)} \int_{0}^{\infty} \frac{u^{-\frac{1}{n}}}{1+u^{2}} d u \\
			& =\frac{1}{n \Gamma\left(1-\frac{1}{n}\right)} \int_{0}^{\frac{\pi}{2}} \tan ^{-\frac{1}{n}}(\theta) d \theta \quad(u=\tan \theta) \\
			& =\frac{1}{n \Gamma\left(1-\frac{1}{n}\right)} \int_{0}^{\frac{\pi}{2}} \sin ^{-\frac{1}{n}}(\theta) \cos \frac{1}{n}(\theta) d \theta \\
			& =\frac{1}{2 n \Gamma\left(1-\frac{1}{n}\right)} \mathrm{B}\left(\frac{1-n}{2},\  \frac{1+n}{2}\right) \\
			& =\frac{1}{2 n \Gamma\left(1-\frac{1}{n}\right)} \Gamma\left(\frac{n-1}{2 n}\right) \Gamma\left(\frac{n+1}{2 n}\right) \\
			& =\frac{\sin \left(\frac{\pi}{n}\right)}{2 n \cos \left(\frac{\pi}{2 n}\right)} \Gamma\left(\frac{1}{n}\right) \\
			& =\frac{1}{n} \sin \left(\frac{\pi}{2 n}\right) \Gamma\left(\frac{1}{n}\right)
		\end{aligned}$$
		法二:
		令$f(z)=e^{-z^n},\ $构造在第一象限角度为$\frac{\pi}{2n}$扇形区域,\ 由留数定理
		$$\int_{0}^{R}\mathrm{e}^{-x^n}\,\mathrm{d}x+\int_{\Gamma_R}\mathrm{e}^{-z^n}\,\mathrm{d}z+\int_{L_2}\mathrm{e}^{-z^n}\,\mathrm{d}z=0,\ $$
		先计算$\int_{L_2}\mathrm{e}^{-z^n}\,\mathrm{d}z,\ $令
		$$z=x\mathrm{e}^{\frac{\pi}{2n}\mathrm{i}},\ z^n=\mathrm{i}x^n,\ \mathrm{d}z=\mathrm{e}^{\frac{\pi}{2n}\mathrm{i}}\,\mathrm{d}x$$
		所以
		$$\int_{L_2}\mathrm{e}^{-z^n}\,\mathrm{d}z=\mathrm{e}^{\frac{\pi}{2n}\mathrm{i}}\int_{R}^{0}e^{-\mathrm{i}x^n}\,\mathrm{d}x=\mathrm{e}^{-\frac{n}{2\pi}\mathrm{i}}\int_{0}^{R}\left[\cos(x^n)-\mathrm{i}\sin(x^n)\right]\,\mathrm{d}x$$
		$$\lim\limits_{R\rightarrow+\infty}\int_{\Gamma_R}\mathrm{e}^{-z^n}\,\mathrm{d}z=\mathrm{i}\frac{\pi}{2n}\lim\limits_{z\rightarrow+\infty}z\mathrm{e}^{-z^n}=0.$$
		最后计算 $\int_{0}^{+\infty} \mathrm{e}^{-x^{n}}\,\mathrm{d} x $\\
		令 $ x^{n}=t,\  x=t^{\frac{1}{n}},\ \mathrm{d} x=\frac{1}{n} t^{\frac{1}{n}-1}\,\mathrm{d} t$ 
		$$\int_{0}^{+\infty} \mathrm{e}^{-x^{n}}\, \mathrm{d} x=\frac{1}{n} \int_{0}^{+\infty} t^{\frac{1}{n}-1} \mathrm{e}^{-t}\,\mathrm{d} t=\frac{1}{n} \Gamma\left(\frac{1}{n}\right)$$
		所以  $R \rightarrow \infty$  时
		$$\begin{aligned}
			\mathrm{e}^{\frac{\pi}{2 n} i} \int_{0}^{+\infty}\left[\cos \left(x^{n}\right)-i \sin \left(x^{n}\right)\right]\,\mathrm{d} x & =\frac{1}{n} \Gamma\left(\frac{1}{n}\right) \\
			\int_{0}^{+\infty} \cos \left(x^{n}\right)\,\mathrm{d} x-i \int_{0}^{+\infty} \sin \left(x^{n}\right)\,\mathrm{d} x=\frac{e^{-\frac{\pi}{2 n} i}}{n} \Gamma\left(\frac{1}{n}\right) & =\frac{\Gamma\left(\frac{1}{n}\right)}{n} \cos \frac{\pi}{2 n}-i \frac{\Gamma\left(\frac{1}{n}\right)}{n} \sin \frac{\pi}{2 n}
		\end{aligned}$$	
		比较实部与虚部,\  得
		$$\begin{array}{l}
			\int_{0}^{+\infty} \cos \left(x^{n}\right)\,\mathrm{d} x=\frac{\Gamma\left(\frac{1}{n}\right)}{n} \cos \frac{\pi}{2 n} \\
			\int_{0}^{+\infty} \sin \left(x^{n}\right)\,\mathrm{d} x=\frac{\Gamma\left(\frac{1}{n}\right)}{n} \sin \frac{\pi}{2 n}
		\end{array}$$
		当  $n=2$  时,\  得到著名的 Fresnel 积分
		$$\int_{0}^{+\infty}\cos x^{2}\,\mathrm{d} x=\int_{0}^{+\infty} \sin x^{2}\,\mathrm{d} x=\frac{1}{2} \sqrt{\frac{\pi}{2}}$$
	\end{solution}
	\newpage
	\begin{problem}
		计算Dirichlet积分
		$$\int_{0}^{\infty}\frac{\sin x}{x}\,\mathrm{d}x$$
	\end{problem}
	\begin{solution}
		1. Fourier 正弦展开
		$$\int_{0}^{\infty} \frac{\sin x}{x}\,\mathrm{d}x=\lim_{m \rightarrow \infty} \int_{0}^{m\pi} \frac{\sin x}{x}\,\mathrm{d}x$$
		$\text{令 } h=\frac{\pi }{k},\  \text{将区间 } [0,\  m\pi]  \text{分割成}km\text{个长度为}h\text{的小区间},\ \text{ 由黎曼积分的定义}$
		$$\int_{0}^{m\pi} \frac{\sin x}{x}\,\mathrm{d}x=\lim\limits_{h \rightarrow 0^{+}} \sum_{n=1}^{km} \frac{\sin nh}{nh}h=\lim\limits_{h \rightarrow 0^{+}} \sum_{n=1}^{km} \frac{\sin nh}{n}$$
		由 Fourier 正弦展开	
		$$\frac{\pi-h}{2}=\sum_{n=1}^{\infty} \frac{\sin nh}{n},\ (0<h<\pi)$$
		所以
		$$\begin{aligned}
			\int_{0}^{\infty} \frac{\sin x}{x}\,\mathrm{d}x & =\lim\limits_{m\rightarrow \infty} \int_{0}^{m\pi} \frac{\sin x}{\mathrm{x}} \,\mathrm{d}x \\
			& =\lim\limits_{m\rightarrow \infty} \lim\limits_{h \rightarrow 0^{+}} \sum_{n=1}^{km} \frac{\sin nh}{n} \\
			& =\lim\limits_{h \rightarrow 0^{+}} \lim\limits_{m\rightarrow \infty} \sum_{n=1}^{km} \frac{\sin nh}{n} \\
			& =\lim\limits_{h\rightarrow 0^{+}} \frac{\pi-h}{2} \\
			& =\frac{\pi}{2}
		\end{aligned}$$
		2. 交换积分次序
		$$\begin{aligned}
			\int_{0}^{\infty} \frac{\sin x}{x}\,\mathrm{d}x & =\int_{0}^{\infty} \sin x \frac{1}{x}\,\mathrm{d}x \\
			& =\int_{0}^{\infty} \sin x\left(\int_{0}^{\infty} \mathrm{e}^{-xy}\,\mathrm{d}y\right)\,\mathrm{d}x\quad(\text { 交换积分次序 }) \\
			& =\int_{0}^{\infty}\left(\int_{0}^{\infty} \mathrm{e}^{yx} \sin x\,\mathrm{d}x\right)\,\mathrm{d}y \\
			\int_{0}^{\infty} \mathrm{e}^{-yx} \sin x\,\mathrm{d}y & =\int_{0}^{\infty} \mathrm{e}^{-yx} \frac{1}{2 \mathrm{i}}\left(\mathrm{e}^{\mathrm{i}x}-\mathrm{e}^{-\mathrm{i}x}\right)\,\mathrm{d}x\\
			& =\frac{1}{2 \mathrm{i}} \int_{0}^{\infty}\left(\mathrm{e}^{-(y-\mathrm{i}) x}-\mathrm{e}^{-(y+\mathrm{i})x}\right)\,\mathrm{d}x \\
			& =\left.\frac{1}{2 \mathrm{i}}\left(-\frac{1}{y-\mathrm{i}} \mathrm{e}^{-(y-\mathrm{i}) x}+\frac{1}{y+\mathrm{i}} \mathrm{e}^{-(y+\mathrm{i})x}\right)\right|_{0} ^{\infty} \\
			& =\frac{1}{2 \mathrm{i}}\left(\frac{1}{y-\mathrm{i}}-\frac{1}{y+\mathrm{i}}\right) \\
			& =\frac{1}{y^{2}+1}
		\end{aligned}$$\\
		故$$\int_{0}^{\infty}\frac{\sin x}{x}\,\mathrm{d}x=\int_{0}^{\infty} \frac{1}{y^{2}+1} \,\mathrm{d}y=\left.\arctan y\right|_{0} ^{\infty}=\frac{\pi}{2} .$$
		
		3. 构造含参变量函数
		
		记 $ I=\int_{0}^{\infty} \frac{\sin t}{t}\,\mathrm{d}t ,\ $ 构造函数$  f(x)=\int_{0}^{\infty} \mathrm{e}^{-xt} \frac{\sin t}{t} \,\mathrm{d}t ,\ $ 则 $ f(0)=I.$
		$$0 \leq|f(x)| \leq \int_{0}^{\infty} \mathrm{e}^{-xt}\left|\frac{\sin t}{t}\right|\,\mathrm{d}t\leq \int_{0}^{\infty} \mathrm{e}^{-xt} \,\mathrm{d}t=\frac{1}{x}$$
		两边取极限,\   $x\rightarrow \infty,\ f(\infty)=0 .$
		$$\begin{aligned}
			f^{\prime}(x) & =\int_{0}^{\infty} \frac{\partial}{\partial x}\left(\mathrm{e}^{-xt} \frac{\sin t}{t}\right)\,\mathrm{d}t \\
			& =-\int_{0}^{\infty} \mathrm{e}^{-xt} \sin t \,\mathrm{d}t \\
			& =-\frac{1}{x^{2}+1} \quad \text { (由上一方法中的结果) }
		\end{aligned}$$
		由牛顿-莱布尼兹公式
		$$0-I=f(\infty)-f(0)=\int_{0}^{\infty}f^{\prime}(x) \mathrm{d}x=-\int_{0}^{\infty} \frac{1}{x^{2}+1} \mathrm{d}=-\left.\arctan x\right|_{0} ^{\infty}=-\frac{\pi}{2}$$
		所以 $ I=\frac{\pi}{2} .$
		
		4. Laplace 变换
		
		令 $ f(t)=\int_{0}^{\infty} \frac{\sin tx}{x}\,\mathrm{d}x,\  t>0 ,\ $ 对 $ f(t)$  作拉普拉斯变换,\  令
		$$\begin{aligned}
			F(x)=\mathscr{L}[f(t)] & =\mathscr{L}\left[\int_{0}^{\infty} \frac{\sin tx}{x} \,\mathrm{d}x\right]_{t} \\
			& =\int_{0}^{\infty} \frac{1}{x} \mathscr{L}[\sin tx]_{t} \,\mathrm{d}x \\
			& =\int_{0}^{\infty} \frac{1}{s^{2}+x^{2}}\,\mathrm{d}x \\
			& =\frac{1}{s \int_{0}^{\infty} \frac{1}{1+\left(\frac{x}{s}\right)^{2}}} \,\mathrm{d}\left(\frac{x}{s}\right) \quad\left(\text { let } u=\frac{x}{s}\right) \\
			& =\left.\frac{1}{s} \arctan u\right|_{0} ^{\infty} \\
			& =\frac{1}{s} \frac{\pi}{2}
		\end{aligned}$$
		则  $$f(t)=\mathscr{L}^{-1}[F(s)]=\frac{\pi}{2} \mathscr{L}^{-1}\left[\frac{1}{s}\right]=\frac{\pi}{2} .$$
		令人惊奇的是,\  $ f(t)  $的值竟与$  t  $无关,\  于是我们得到一个更为普遍的结论
		$$\int_{0}^{\infty} \frac{\sin tx}{x} \,\mathrm{d}x=\frac{\pi}{2},\  t>0$$
		$$\begin{aligned}
			f^{\prime}(t) & =\int_{0}^{\infty} \frac{\partial}{\partial t}\left(\frac{\sin tx}{x}\right)\,\mathrm{d}x\\
			& =\int_{0}^{\infty} \cos tx\,\mathrm{d}x \\
			& =\lim\limits_{n\rightarrow \infty} \int_{0}^{n\pi / t} \cos tx \mathrm{dx} \quad(\operatorname{let} u=tx) \\
			& =\lim\limits_{n \rightarrow \infty} \frac{1}{t} \int_{0}^{\mathrm{n} \pi} \cos u\,\mathrm{d}u \\
			& =\left.\frac{1}{t} \lim\limits_{n\rightarrow \infty} \sin u\right|_{0} ^{n\pi} \\
			& =0
		\end{aligned}$$
		所以  $f(t)=C ,\  $与 $ t $ 无关.
		
		5. Fourier 变换
		
		令  $f(t)=\left\{\begin{array}{l}
			1,\ |t|<1 \\
			0,\ |t| \geq 1
		\end{array}\right. ,\ $ 对 $ f(t) $ 作傅里叶变换,\  令
		$$\begin{aligned}
			F(\mu) & =\mathscr{F}[f(t)] \\
			& =\int_{-\infty}^{\infty} f(t) \mathrm{e}^{-\mathrm{i} \mu t} \,\mathrm{d}t \\
			& =\int_{-1}^{1} \mathrm{e}^{-\mathrm{i} \mu t} \,\mathrm{d}t\\
			& =-\left.\frac{1}{\mathrm{i} \mu} \mathrm{e}^{-\mathrm{i} \mu t}\right|_{-1} ^{1} \\
			& =\frac{2}{\mu} \frac{1}{2 \mathrm{i}}\left(\mathrm{e}^{\mathrm{i} \mu}-\mathrm{e}^{-\mathrm{i} \mu}\right) \\
			& =2 \frac{\sin \mu}{\mu}
		\end{aligned}$$
		则  $$f(t)=\mathscr{F}^{-1}[F(\mu)]=\frac{1}{2 \pi} \int_{-\infty}^{\infty} 2 \frac{\sin \mu}{\mu} \mathrm{e}^{\mathrm{i} \mu t}\,\mathrm{d} \mu .$$
		取 $ t=0 ,\ $ 则
		$$1=f(0)=\frac{1}{\pi} \int_{-\infty}^{\infty} \frac{\sin \mu}{\mu} \,\mathrm{d} \mu=\frac{2}{\pi} \int_{0}^{\infty} \frac{\sin \mu}{\mu} \,\mathrm{d}\mu$$
		故  $$\int_{0}^{\infty} \frac{\sin \mu}{\mu}\,\mathrm{d} \mu=\frac{\pi}{2} . $$
		6. 狄拉克函数
		
		首先来介绍一下狄拉克函数(就是 Dirac 创造的函数),\  也称脉冲函数(比较形象):
		$$\delta(t)=\left\{\begin{array}{ll}
			\infty,\  & t=0 \\
			0,\  & t \neq 0
		\end{array},\ \text {且满足 } \int_{-\infty}^{\infty} \delta(t) \,\mathrm{d}t=1 .\right.$$
		容易验证:$  \int_{-\infty}^{\infty} \delta(t) \mathrm{f}(\mathrm{t}) \,\mathrm{d}t=f(0) .$\\
		取  $f(t)=\mathrm{e}^{-\mathrm{i} \mu t},\  f(0)=1 .$\\
		则脉冲函数的傅里叶变换  $F(\mu)=\mathscr{F}[\delta(t)]=\int_{-\infty}^{\infty} \delta(t) \mathrm{e}^{-\mathrm{i} \mu t} \,\mathrm{d}t=1 .$\\
		作傅里叶反变换  $\delta(t)=\mathscr{F}^{-1}[F(\mu)]=\frac{1}{2 \pi} \int_{-\infty}^{\infty} \mathrm{e}^{\mathrm{i} \mu t}\,\mathrm{d}\mu .$\\
		准备工作完成,\  构造函数 $ g(\lambda)=\int_{-\infty}^{\infty} \frac{\sin \lambda x}{x} \,\mathrm{d}x ,\  则  g(1)=\int_{-\infty}^{\infty} \frac{\sin x}{x} \,\mathrm{d}x .$\\
		$$\begin{aligned}
			g^{\prime}(\lambda) & =\int_{-\infty}^{\infty} \frac{\partial}{\partial \lambda}\left(\frac{\sin \lambda x}{x}\right)\,\mathrm{d}x\\
			& =\int_{-\infty}^{\infty} \cos \lambda x\,\mathrm{d}x+0 \\
			& =\int_{-\infty}^{\infty} \cos \lambda x\,\mathrm{d}x+\mathrm{i} \int_{-\infty}^{\infty} \sin \lambda x\,\mathrm{d}x \\
			& =\int_{-\infty}^{\infty} \mathrm{e}^{\mathrm{i} \lambda x} \,\mathrm{d}x \\
			& =2 \pi\left(\frac{1}{2 \pi} \int_{-\infty}^{\infty} \mathrm{e}^{\mathrm{i} \lambda x} \,\mathrm{d}x\right) \\
			& =2 \pi \delta(\lambda)
		\end{aligned}$$
		因为  $g(\lambda)$  是奇函数,\  所以
		$$\begin{aligned}
			g(1) & =-g(-1) \\
			& =\frac{1}{2}(g(1)-g(-1)) \quad(\text { 由 N-L 公式 }) \\
			& =\frac{1}{2} \int_{-1}^{1} g^{\prime}(\lambda)\,\mathrm{d} \lambda \\
			& =\frac{1}{2} \int_{-1}^{1} 2 \pi \delta(\lambda)\,\mathrm{d} \lambda \quad(\delta(\lambda)=0,\  \lambda \neq 0) \\
			& =\pi \int_{-\infty}^{\infty} \delta(\lambda)\,\mathrm{d}\lambda \\
			& =\pi
		\end{aligned}$$
		7. 留数定理
		
		定理内容:当被积函数 $ f(\mathrm{x}) $ 是  $x  $的有理函数 (多项式除多项式),\  且分母的次数比分子 的次数至少高一次,\   $f(z) $ 在实轴上除去有限多个一级奇点$ x_{1},\  x_{2},\  \cdots,\  x_{p}  $外处处解析,\  在上半复平面$  (\operatorname{Im} z>0) $ 除去有限多个奇点$z_{1},\  z_{2},\  \cdots,\ z_{q} $ 外处处解析,\  则
		$$\int_{-\infty}^{\infty} f(x) \mathrm{e}^{\mathrm{i}mx}\,\mathrm{d}x=\pi \mathrm{i} \sum_{k=1}^{p} \operatorname{Res}\left[f(z) \mathrm{e}^{\mathrm{i}mz},\  x_{k}\right]+2 \pi \mathrm{i} \sum_{k=1}^{q} \operatorname{Res}\left[f(z) \mathrm{e}^{\mathrm{i}mz},\  z_{k}\right]$$
		其中 $ \operatorname{Res}\left[f(z),\ z_{0}\right] $ 为函数 $ f $ 在 $ z_{0}  $处的留数,\  定义如下:
		若$z_{0}  $是  $f(\mathrm{z})$  的孤立奇点,\   $f(z) $ 在$  D=\left\{z|0<| z-z_{0} \mid<R\right\} $ 内解析,\  $C $是 $ D $ 内包 围 $ z_{0} $ 的任一正向简单闭曲线,\  则称积分
		$$\frac{1}{2 \pi \mathrm{i}} \oint_{C} f(z)\,\mathrm{d}z$$
		为$ f$ 在$ z_{0}$  处的留数,\  记作  $\operatorname{Res}\left[f(z),\  z_{0}\right] .$
		套用定理,\  令  $f(x)=\frac{1}{x} ,\ $ 实轴上的一级奇点  $x_{1}=0 ,\  $上半复平面内无奇点,\  则
		$$\begin{array}{l}
			\int_{-\infty}^{\infty} \frac{1}{x} e^{\mathrm{i}x} \,\mathrm{d}x=\mathrm{i} \pi \operatorname{Res}\left[\frac{\mathrm{e}^{\mathrm{i}z}}{z},\  0\right]=\mathrm{i} \pi \frac{1}{2 \pi \mathrm{i}} \oint_{|z|-1} \frac{\mathrm{e}^{\mathrm{i}z}}{z}\,\mathrm{d}z=\frac{1}{2} \oint_{|z|-1} \frac{\mathrm{e}^{\mathrm{i}z}}{z}\,\mathrm{d}z \\
			\oint_{|z|-1} \frac{e^{\mathrm{i}z}}{z} \,\mathrm{d}z=\oint_{|\mathrm{i}z|-1} \frac{\mathrm{e}^{\mathrm{i}z z}}{\mathrm{i}z}\,\mathrm{d}(\mathrm{i}z) \\
			=\oint_{|z|-1} \frac{\mathrm{e}^{z}}{z} \,\mathrm{d}z \\
			=\oint_{|z|-1} \frac{1}{z}\left(1+z+\frac{z^{2}}{2 !}+\cdots+\frac{z^{n}}{n !}+\cdots\right) d z \\
			=\oint_{|z|-1}\left(\frac{1}{z}+1+\frac{z}{2 !}+\cdots+\frac{z^{n}}{(n+1)!}+\cdots\right) \,\mathrm{d}z \\
			=\oint_{|z|-1} \frac{1}{z} \,\mathrm{d} z+\oint_{|z|-1}\left(\mathrm{d}(z)+\frac{\mathrm{d}\left(z^{2}\right)}{2 \cdot 2 !}+\cdots+\frac{\mathrm{d}\left(z^{n+1}\right)}{(n+1)(n+1) !}+\cdots\right) \\
			=\oint_{|z|-1} \frac{1}{z}\,\mathrm{d}z \\
		\end{array}$$
		三角换元,\  令 $z=\mathrm{e}^{\mathrm{i} \theta}(0 \leq \theta \leq 2 \pi) ,\ $ 则  $\frac{\mathrm{d}z}{\mathrm{d} \theta}=\mathrm{ie}^{\mathrm{i} \theta}=\mathrm{i}z,\  \frac{\mathrm{d}z}{z}=\mathrm{i}\,\mathrm{d} \theta .$
		$$\oint_{|z|-1} \frac{1}{z}\,\mathrm{d}z=\int_{0}^{2 \pi} \mathrm{i}\,\mathrm{d} \theta=2 \pi \mathrm{i}$$
		于是
		$$\int_{-\infty}^{\infty} \frac{1}{x} e^{\mathrm{i}x} \,\mathrm{d}x=\frac{1}{2} \oint_{|z|-1} \frac{\mathrm{e}^{\mathrm{i} z}}{z} \,\mathrm{d}z=\frac{1}{2} \oint_{|z|-1} \frac{1}{z} \mathrm{dz}=\pi \mathrm{i}$$
		又因为
		$$\int_{-\infty}^{\infty} \frac{1}{x} \mathrm{e}^{\mathrm{i} x}\,\mathrm{d} x=\int_{-\infty}^{\infty} \frac{1}{x}(\cos x+i \sin x)\,\mathrm{d} x=\mathrm{i} \int_{-\infty}^{\infty} \frac{\sin x}{x} \,\mathrm{d}x$$
		故
		$$\int_{-\infty}^{+\infty}\frac{\sin x}{x}\,\mathrm{d}x=\pi.$$
		8. 黎曼引理
		
		先做些准备工作
		$$\sin \frac{2 n+1}{2}x=\sin \frac{x}{2}+\sum_{k=1}^{n}\left(\sin \frac{2 k+1}{2} x-\sin \frac{2k-1}{2}x\right)$$
		由和差化积公式:  $\sin A-\sin B=2 \sin \frac{A-B}{2} \cos \frac{A+B}{2} .$\\
		则$  \sin \frac{2 n+1}{2} x=\left(\frac{1}{2}+\sum_{k=1}^{n} \cos kx\right) 2 \sin \frac{x}{2} .$\\
		$x\neq 2k\pi $ 时,\  有 $ \frac{\sin \frac{2 n+1}{2}x}{2 \sin \frac{x}{2}}=\frac{1}{2}+\sum_{k=1}^{n} \cos kx .$\\
		两边同时积分,\  得 $ \int_{0}^{\pi} \frac{\sin \frac{2 n+1}{2}x}{2 \sin \frac{x}{2}}=\frac{\pi}{2}(n=0,\ 1,\ 2,\  \cdots) .$
		$$\text { 令 } g(x)=\frac{1}{x}-\frac{1}{2 \sin \frac{x}{2}}=\frac{2 \sin \frac{x}{2}-x}{2 x \sin \frac{x}{2}},\  0<x \leq \pi \text {. }$$
		由洛必达法则,\ 
		$$\lim\limits_{x \rightarrow 0^{+}} g(x)=\lim\limits_{x \rightarrow 0^{+}} \frac{\cos \frac{x}{2}-1}{x \cos \frac{x}{2}+2 \sin \frac{x}{2}}=\lim\limits_{x \rightarrow 0^{+}} \frac{-\frac{1}{2} \sin \frac{x}{2}}{2 \cos \frac{x}{2}-\frac{1}{2} x \sin \frac{x}{2}}=0$$
		祈充定义 $ g(0)=0 ,\  $则  $g$ 在$  [0,\  \pi]  $上连续.\\
		Riemann-Lebesgue引理: 若 $f$  在$  [a,\ b]  $上连续,\  则  $\lim\limits_{p \rightarrow \infty} \int_{a}^{b}f(x) \sin px\,\mathrm{d}x=0 .$\\
		令  $f(x)=g(x),\  p=n+\frac{1}{2} ,\  $则
		$$\lim\limits_{n \rightarrow \infty} \int_{0}^{\pi}\left(\frac{1}{x}-\frac{1}{2 \sin \frac{x}{2}}\right) \sin \left(n+\frac{1}{2}\right) x\,\mathrm{d} x=0$$
		$$\lim\limits_{n \rightarrow \infty} \int_{0}^{\pi} \frac{\sin \left(n+\frac{1}{2}\right) x}{x} \,\mathrm{d} x=\lim\limits_{n \rightarrow \infty} \int_{0}^{\pi} \frac{\sin \left(\frac{2 n+1}{2}\right) x}{2 \sin \frac{x}{2}} \,\mathrm{d}x=\lim\limits_{n \rightarrow \infty} \frac{\pi}{2}=\frac{\pi}{2}$$
		令  $u=\left(n+\frac{1}{2}\right)x,\  $则
		$$\lim\limits_{n \rightarrow \infty} \int_{0}^{\pi} \frac{\sin \left(n+\frac{1}{2}\right) x}{x} d x=\lim\limits_{n \rightarrow \infty} \int_{0}^{\left(n+\frac{1}{2}\right) \pi} \frac{\sin u}{u}\,\mathrm{d} u=\frac{\pi}{2}$$
		所以
		$$\int_{0}^{\infty} \frac{\sin u}{u} d u=\lim\limits_{n \rightarrow \infty} \int_{0}^{\left(n+\frac{1}{2}\right) \pi} \frac{\sin u}{u} d u=\frac{\pi}{2}$$
	\end{solution}
	\newpage
	\begin{problem}
		计算Laplace积分:
		$$I_1=\int_{0}^{+\infty}\frac{\cos bx}{x^2+a^2}\,\mathrm{d}x,\ \quad I_2=\int_{0}^{+\infty}\frac{x\sin bx}{x^2+a^2}\,\mathrm{d}x,\ a,\ b>0$$
	\end{problem}
	\begin{solution}
		\begin{align*}
			I_1'(b)&=-\int_{0}^{+\infty}\frac{x\sin bx}{x^2+a^2}\,\mathrm{d}x\\
			&=-\int_{0}^{+\infty}\frac{(a^2+x^2-a^2)\sin bx}{x(x^2+a^2)}\,\mathrm{d}x\\
			&=-\int_{0}^{+\infty}\frac{\sin bx}{x}\,\mathrm{d}x+a^2\int_{0}^{+\infty}\frac{\sin bx}{x(x^2+a^2)}\,\mathrm{d}x\\
			&=-\frac{\pi}{2}+a^2\int_{0}^{+\infty}\frac{\sin bx}{x(x^2+a^2)}\,\mathrm{d}x\\
			I_1''(b)&=a^2\int_{0}^{+\infty}\frac{\cos bx}{x^2+a^2}\,\mathrm{d}x=a^2I_1(b)
		\end{align*}
	这对应一个二阶微分方程$I_1''(b)-a^2I_1(b)=0$\\
	初值问题为
	$$I_1(0)=\int_{0}^{+\infty}\frac{1}{x^2+a^2}\,\mathrm{d}x=\frac{\pi}{2a},\ I_1'(0)=-\frac{\pi}{2}$$
	可解得$I_1(b)=\frac{\pi}{2a}\mathrm{e}^{-ab}$\\
	又有$I_2=-I_1'(b)=\frac{\pi}{2}\mathrm{e}^{-ab}.$\\
%	法二、二重积分:\\
%	首先证明
%	$$I_3=I_3(b)=\int_{0}^{+\infty}\mathrm{e}^{-ax^2}\cos bx\,\mathrm{d}x=\frac{1}{2}\sqrt{\frac{\pi}{a}}\mathrm{e}^{-\frac{b^2}{4a}}$$
%	与法一同理:
%	\begin{align*}
%		I_3'(b)&=-\int_{0}^{+\infty}x\mathrm{e}^{-ax^2}\sin bx\,\mathrm{d}x\\
%		&=\frac{1}{2a}\int_{0}^{+\infty}\sin bx\,\mathrm{d}(\mathrm{e}^{-ax^2})\\
%		&=\left.\frac{1}{2a}\mathrm{e}^{-ax^2}\sin bx\right|_{0}^{+\infty}-\frac{b}{2a}\int_{0}^{+\infty}\mathrm{e}^{-ax^2}
%		\cos bx\,\mathrm{d}x\\
%		&=-\frac{b}{2a}I_3(b)
%	\end{align*}
%	这对应一个一阶微分方程$I_3'(b)+\frac{b}{2a}I_3(b)=0.$
%	初值问题为
%	$$I_3(0)=\int_{0}^{+\infty}\mathrm{e}^{-ax^2}\,\mathrm{d}x=\frac{1}{\sqrt{a}}\int_{0}^{+\infty}\mathrm{e}^{-t^2}\,\mathrm{d}t=\frac{1}{2}\sqrt{\frac{\pi}{a}}.$$
%	可解得$I_3(b)=\frac{1}{2}\sqrt{\frac{\pi}{a}}\mathrm{e}^{-\frac{b^2}{4a}}.$\\
%	与法一同理可求得$I_4(b)=I_3'(b)=\frac{b}{4a}\sqrt{\frac{\pi}{a}}\mathrm{e}^{-\frac{b^2}{4a}}$\\
%	一个引理:$\int_{-\infty}^{+\infty}f(bx-\frac{a}{x})\,\mathrm{d}x=\frac{1}{b}\int_{-\infty}^{+\infty}f(x)\,\mathrm{d}x$\\
%	注意到$\int_{0}^{+\infty}\mathrm{e}^{-y(x^2+a^2)}\,\mathrm{d}y=\frac{1}{x^2+a^2}.$所以有
%	\begin{align*}
%		I_1&=\int_{0}^{+\infty}\frac{\cos bx}{x^2+a^2}\,\mathrm{d}x\\
%		&=\int_{0}^{+\infty}\cos bx\,\mathrm{d}x\int_{0}^{+\infty}\mathrm{e}^{-y(x^2+a^2)}\,\mathrm{d}y\\
%		&=\\
%	\end{align*}
	
	\end{solution}
	\newpage
	\begin{problem}
		已知数列$\{a_n\}$满足$\frac{a_n-\tan^2 n}{\tan 2}=(a_n+1)\tan n,\ $计算其前$n$项和$S_n.$
	\end{problem}
	\begin{solution}
		由已知条件可得$a_n-\tan^2n=a_n\tan n\tan 2+\tan n\tan 2,\ $所以
		$$a_n=\frac{\tan^2n+\tan n\tan 2}{1-\tan n\tan 2}=\frac{\tan n+\tan 2}{1-\tan n\tan 2}\cdot\tan n$$
		由两角和正切公式可知
		$$\frac{\tan n+\tan 2}{1-\tan n\tan 2}=\tan (n+2),\ $$
		所以$a_n=\tan(n+2)\tan n.$\\
		由于
		$$\tan 2=\tan (n+2-n)=\frac{\tan(n+2)-\tan n}{1+\tan(n+2)\tan n},\ $$
		所以
		$$a_n=\tan(n+2)\tan n=\frac{\tan(n+2)-\tan n}{\tan 2}-1.$$
		进一步有
		$$\begin{aligned}
			S_n&=\frac{\tan 3-\tan 1}{\tan 2}-1+\frac{\tan 4-\tan 2}{\tan 2}-1+\frac{\tan 5-\tan 3}{\tan 2}-1+\cdots\\
			&+\frac{\tan(n+1)-\tan(n-1)}{\tan 2}-1+\frac{\tan(n+2)-\tan n}{\tan 2}-1\\
			&=\frac{\tan(n+1)+\tan(n+2)-\tan 1-\tan 2}{\tan 2}-n\\
			&=\frac{\tan(n+1)+\tan(n+2)-\tan 1}{\tan 2}-n-1.
		\end{aligned}$$
	\end{solution}
	\newpage
	\begin{problem}
		求极限
		$$\lim\limits_{x\rightarrow 1}\left(\frac{m}{1-x^m}-\frac{n}{1-x^n}\right)$$
	\end{problem}
	\begin{solution}
		\begin{align*}
			\lim\limits_{x\rightarrow 1}\frac{m}{1-x^m}-\frac{n}{1-x^n}&=\lim\limits_{x\rightarrow 1}\frac{m(1-x^n)-n(1-x^n)}{(1-x^m)(1-x^n)}\\
			&=\lim\limits_{x\rightarrow 1}\frac{m\sum\limits_{k=0}^{n-1}x^k-n\sum\limits_{k=0}^{m-1}x^k}{\left(\sum\limits_{k=0}^{n-1}x^k\right)\cdot\left(\sum\limits_{k=0}^{n-1}x^k\right)(1-x)}\\
			&=\frac{1}{mn}\cdot\lim\limits_{x\rightarrow 1}\left[\frac{m\sum\limits_{k=0}^{n-1}x^k-n\sum\limits_{k=0}^{m-1}x^k}{(1-x)}\right]\\
			&=\frac{1}{mn}\cdot\lim\limits_{x\rightarrow 1}\left[\frac{m\sum\limits_{k=0}^{n-2}(k+1)x^k-n\sum\limits_{k=0}^{m-2}(k+1)x^k}{-1}\right](L'Hospital)\\
			&=\frac{1}{mn}\left[\frac{m\frac{n(n-1)}{2}-n\frac{m(m-1)}{2}}{-1}\right]\\
			&=\frac{m-n}{2}
		\end{align*}
	\end{solution}
	\newpage
	\begin{problem}
		设$f(x)$在$[a,\ b]$上可积,\ $g(x)\geqslant 0,\ g$是以$T>0$为周期的函数,\ 在$[0,\ T]$上可积,\ 试证$$\lim\limits_{n\to\infty}\int_{a}^{b}f(x)g(nx)\,\text{d}x=\frac{1}{T}\int_{0}^{T}g(x)\,\text{d}x\int_{a}^{b}f(x)\,\text{d}x.$$
	\end{problem}
	\begin{proof}
		因$g(x)$以$T$为周期,\ 因此$g(nx)$以$\frac{T}{n}$为周期,\ 当$n$充分大时,\ $[a,\ b]$含有$g(nx)$的多个周期,\ 为了把区间变成$\frac{T}{n}$的整数倍,\ 取足够大的正整数$m,\ $使得
		$[A,\  B]=[-m T,\  m T] \supset[a,\  b] . $这时  $[A,\  B]  $相当  $g(n x) $ 的 $ 2 m n $ 个周期. 令
		$$F(x)=\left\{\begin{array}{l}
			f(x),\  x \in[a,\  b],\  \\
			0,\  \quad x \in[A,\  B] \backslash[a,\  b],\ 
		\end{array}\right.$$
		于是  $F(x)  $在 $ [A,\  B]  $上可积,\  且
		$$I_{n} \equiv \int_{a}^{b} f(x) g(n x) \,\mathrm{d} x=\int_{a}^{B} F(x) g(n x) \,\mathrm{d} x .$$
		将  $[A,\  B] 2 m n$  等分,\ 作分划  $A=x_{0}<x_{1}<\cdots<x_{2 m n}=B . $每个小区间恰是  $g(n x)  $的一个 周期,\ 小区间长度等于$  \frac{T}{n} .$ 于是
		$$I_{n}=\int_{A}^{B} F(x) g(n x) \,\mathrm{d} x=\sum_{i=1}^{2 m n} \int_{x_{i-1}}^{x_{i}} F(x) g(n x) \,\mathrm{d} x .$$
		注意到 $ g(x) \geqslant 0 ,\ $应用第一积分中值定理,\ 得
		$$I_{n}=\sum_{i=1}^{2 m n} c_{i} \int_{x_{i-1}}^{x_{i}} g(n x) \,\mathrm{d} x,\ $$
		其中$ c_{i}: m_{i} \equiv \inf\limits_{x_{i-1} \leqslant x \leqslant x_{i}} F(x) \leqslant c_{i} \leqslant M_{i} \equiv \sup \limits_{x_{i-1} \leqslant x \leqslant x_{i}} F(x) . $因  $\left[x_{i-1},\  x_{i}\right] $ 是 $ g(n x) $ 的一个周期,\  令$  n x=t ,\ $ 则
		$$\int_{x_{i-1}}^{x_{i}} g(n x) \,\mathrm{d} x=\int_{0}^{\frac{T}{n}} g(n x) \,\mathrm{d} x=\frac{1}{n} \int_{0}^{T} g(t) \,\mathrm{d} t .$$
		代人上式,\  得
		$$	I_{n}=\frac{1}{T} \int_{0}^{T} g(x) \,\mathrm{d} x \cdot \sum_{i=1}^{2 m n} c_{i} \frac{T}{n}$$
		注意 :	
		$$\sum_{i=1}^{2 m n} m_{i} \frac{T}{n} \leqslant \sum_{i=1}^{2 m n} c_{i} \frac{T}{n} \leqslant \sum_{i=1}^{2 m n} M_{i} \frac{T}{n}$$
		其左、右两端分别为 $ F(x)$  在 $ [A,\  B]$  上的 Darboux 和. 故
		$$\lim _{n \rightarrow \infty} I_{n}=\frac{1}{T} \int_{0}^{T} g(x) \,\mathrm{d} x \cdot \int_{A}^{B} F(x) \,\mathrm{d} x=\frac{1}{T} \int_{0}^{T} g(x) \,\mathrm{d} x \cdot \int_{a}^{b} f(x) \,\,\mathrm{d} x .$$
	\end{proof}
	\newpage
	\begin{problem}
		设$3$阶是对称矩阵$A$的每行元素之和为$3,\ $且$ r(\boldsymbol{A}),\ \beta=(-1,\ 2,\ 2)',\ $求$\boldsymbol{A}^n\boldsymbol{\beta},\ \left(\boldsymbol{A}-\frac{3}{2}I\right)^{10}$
	\end{problem}
	\begin{solution}
		(1) 由已知 $ \boldsymbol{A}$  的特征值 $ \lambda_{1}=3 ,\ $ 其对应的一个特征向量为 $ \boldsymbol{\alpha}_{1}=(1,\ 1,\ 1)^{\mathrm{T}} ,\  $又由 $ r(\boldsymbol{A})=1 ,\  $且 $ \boldsymbol{A} $ 可相似对 角化知  $\boldsymbol{A}  $有二重特征值$  \lambda_{2}=\lambda_{3}=0 .$ 设其对应的特征向量为  $\boldsymbol{x}=\left(x_{1},\  x_{2},\  x_{3}\right)^{\top} ,\  $于是有 $ \left(\boldsymbol{x},\  \boldsymbol{\alpha}_{1}\right)=0 ,\ $ 即 $ x_{1}+x_{2}+x_{3}=0 ,\ $ 解得  $\lambda_{2}=\lambda_{3}=0  $对应的特征向量为
		$$\boldsymbol{x}=k_{1}\left[\begin{array}{c}-1 \\ 1 \\ 0\end{array}\right]+k_{2}\left[\begin{array}{c}-1 \\ 0 \\ 1\end{array}\right],\  k_{1},\  k_{2}\text{为不同时为} 0 \text{的任意常数}.$$  
		取  $\boldsymbol{\alpha}_{2}=(-1,\ 1,\ 0)^{\mathrm{T}},\  \boldsymbol{\alpha}_{3}=(-1,\ 0,\ 1)^{\mathrm{T}} ,\ $ 显然  $\boldsymbol{\alpha}_{1},\  \boldsymbol{\alpha}_{2},\  \boldsymbol{\alpha}_{3} $ 线性无关,\ 于是 $ \boldsymbol{\beta} $ 可由  $\boldsymbol{\alpha}_{1},\  \boldsymbol{\alpha}_{2},\  \boldsymbol{\alpha}_{3}$  线性表示,\  即  $x_{1} \boldsymbol{\alpha}_{1}+x_{2} \boldsymbol{\alpha}_{2}+   x_{3} \boldsymbol{\alpha}_{3}=\boldsymbol{\beta} ,\ $解得 $ x_{1}=1,\  x_{2}=1,\  x_{3}=1 .$
		故  $$\boldsymbol{\alpha}_{1}+\boldsymbol{\alpha}_{2}+\boldsymbol{\alpha}_{3}=\boldsymbol{\beta} ,\  $$所以
		$$\begin{aligned}
			\boldsymbol{A}^{n} \boldsymbol{\beta} & =\boldsymbol{A}^{n}\left(\boldsymbol{\alpha}_{1}+\boldsymbol{\alpha}_{2}+\boldsymbol{\alpha}_{3}\right)=\boldsymbol{A}^{n} \boldsymbol{\alpha}_{1}+\boldsymbol{A}^{n} \boldsymbol{\alpha}_{2}+\boldsymbol{A}^{n} \boldsymbol{\alpha}_{3}=\lambda_{1}^{n} \boldsymbol{\alpha}_{1}+\lambda_{2}^{n} \boldsymbol{\alpha}_{2}+\lambda_{3}^{n} \boldsymbol{\alpha}_{3}=3^{n} \boldsymbol{\alpha}_{1} \\
			& =\left(3^{n},\  3^{n},\  3^{n}\right)^{\mathrm{T}} .
		\end{aligned}$$
		(2) 由于 $ \boldsymbol{A}  $为实对称矩阵,\  所以令  $P=\left(\boldsymbol{\alpha}_{1},\  \boldsymbol{\alpha}_{2},\  \boldsymbol{\alpha}_{3}\right) ,\ $ 则 $ P$  为可逆矩阵.
		且  $P^{-1} A P=\boldsymbol{A} .$ 于是
		$$\begin{aligned}
			\left(\boldsymbol{A}-\frac{3}{2} I\right)^{10} & =\left(P \Lambda P^{-1}-\frac{3}{2} P P^{-1}\right)^{10}=\left[P\left(\Lambda-\frac{3}{2} E\right) P^{-1}\right]^{10}=P\left(\Lambda-\frac{3}{2} E\right)^{10} P^{-1} \\
			& =P\left[\begin{array}{c}
				\left(\frac{3}{2}\right)^{10} \\
				\left(-\frac{3}{2}\right)^{10} \\
				\left(-\frac{3}{2}\right)^{10}
			\end{array}\right] P^{-1}=\left(\frac{3}{2}\right)^{10} P P^{-1}=\left(\frac{3}{2}\right)^{10} E .
		\end{aligned}$$
	\end{solution}
	\newpage
	\begin{problem}
		设函数$f(x)$在$[a,\ b]$上二节可导,\ 且$f''(x)>0.$证明:
		\begin{enumerate}
			\item $$f\left(\frac{a+b}{2}\right)\leqslant\frac{1}{b-a}\int_{a}^{b}f(x)\,\text{d}x;$$
			\item 若$f(x)\leqslant 0,\ x\in[a,\ b],\ $则有$$f(x)\geqslant\frac{2}{b-a}\int_{a}^{b}f(x)\,\text{d}x,\ x\in[a,\ b].$$
		\end{enumerate}
	\end{problem}
	\begin{proof}
		(1) 因  $f^{\prime \prime}(x) \geqslant 0 ,\ $ 故  $f $ 为一凸函数. 取切点为  $\left(\frac{a+b}{2},\  f\left(\frac{a+b}{2}\right)\right) ,\ $ 则必有
		$$f(x) \geqslant f\left(\frac{a+b}{2}\right)+f^{\prime}\left(\frac{a+b}{2}\right)\left(x-\frac{a+b}{2}\right) .$$
		利用积分不等式性质即得
		$$\begin{aligned}
			\int_{a}^{b} f(x) \,\mathrm{d} x & \geqslant f\left(\frac{a+b}{2}\right)(b-a)+f^{\prime}\left(\frac{a+b}{2}\right) \int_{a}^{b}\left(x-\frac{a+b}{2}\right) \,\mathrm{d} x \\
			& =f\left(\frac{a+b}{2}\right)(b-a)+0=f\left(\frac{a+b}{2}\right)(b-a) .
		\end{aligned}$$
		这就证得结论.\\
		(2) 由 $ f $ 为一凸函数,\  $ \forall x,\  t \in[a,\  b] ,\  $有
		$$f(x) \geqslant f(t)+f^{\prime}(t)(x-t) .$$
		以 $t$  作为积分变量,\ 得积分不等式:
		$$\int_{a}^{b} f(x) \,\mathrm{d} t \geqslant \int_{a}^{b} f(t) \,\mathrm{d} t+x \int_{a}^{b} f^{\prime}(t) \,\mathrm{d} t-\int_{a}^{b} t f^{\prime}(t) \,\mathrm{d} t$$
		通过计算相继得到	
		$$\begin{aligned}
			f(x)(b-a) & \geqslant \int_{a}^{b} f(t) \,\mathrm{d} t+x[f(b)-f(a)]-\left.(t f(t))\right|_{a} ^{b}+\int_{a}^{b} f(t) \,\mathrm{d} t,\  \\
			f(x)(b-a) & \geqslant \int_{a}^{b} f(t) \,\mathrm{d} t+x[f(b)-f(a)]-[b f(b)-a f(a)]+\int_{a}^{b} f(t) \,\mathrm{d} t,\  \\
			2 \int_{a}^{b} f(t) \,\mathrm{d} t & \leqslant f(x)(b-a)+(b-x) f(b)+(x-a) f(a) .
		\end{aligned}$$
		因为$  f(x) \leqslant 0,\  x \in[a,\  b] ,\ $ 而 且  $\int_{a}^{b} f(t) \,\mathrm{d} t=\int_{a}^{b} f(x) \,\mathrm{d} x ,\  $所以又得
		$$(b-x) f(b)+(x-a) f(a) \leqslant 0,\ $$
		$$2 \int_{a}^{b} f(x) \,\mathrm{d} x \leqslant f(x)(b-a)+(b-x) f(b)+(x-a) f(a) \leqslant f(x)(b-a),\ $$
		这就证得$$ f(x) \geqslant \frac{2}{b-a} \int_{a}^{b} f(x) \,\mathrm{d} x,\  x \in[a,\  b] .$$
	\end{proof}
	\newpage
	\begin{theorem}
		设 $ X$  为赋范线性空间,则 $ X $ 中开球的闭包  $\overline{B\left(x_{0},\  r\right)}  $等于闭球 $ \bar{B}\left(x_{0},\  r\right) .$
	\end{theorem}
	\begin{proof}
		显然有  $B\left(x_{0},\  r\right) \subset \bar{B}\left(x_{0},\  r\right),\ $距离空间中的开球为开集,\ 闭球为闭集,\  $ \bar{B}\left(x_{0},\  r\right)  $为闭集,从而
		$$\overline{B\left(x_{0},\  r\right)} \subset \overline{\left(\bar{B}\left(x_{0},\  r\right)\right)}=\bar{B}\left(x_{0},\  r\right)$$
		(对于一般的距离空间上式亦成立)
		任取 $ x \in \bar{B}\left(x_{0},\  r\right) ,\ $ 定义  $x_{n}=\frac{n}{n+1}\left(x-x_{0}\right)+x_{0} ,\ $ 则
		$$\left\|x_{n}-x_{0}\right\|=\frac{n}{n+1}\left\|x-x_{0}\right\| \leq \frac{n r}{n+1}<r,\ $$
		从而 $ x_{n} \in B\left(x_{0},\  r\right) .$ 令  $n \rightarrow \infty ,\  $有
		$$\left\|x_{n}-x\right\|=\frac{1}{n+1}\left\|x_{0}-x\right\| \rightarrow 0,\ $$
		从而  $$x \in \overline{B\left(x_{0},\  r\right)},\  \bar{B}\left(x_{0},\  r\right) \subset \overline{B\left(x_{0},\  r\right)} .$$
	\end{proof}
	\begin{note}
		对于一般的距离空间,结论不成立.
		\begin{example}
			考虑距离子空间
			$$X=\left\{x=(\xi,\  \eta) \mid-\frac{1}{2} \leq \xi \leq \frac{1}{2},\  \eta=0\right\} \cup\{(0,\ 1)\} \subset \mathbb{R}^{2},\ $$
			显然 $ X $ 不为线性空间. 由
			$$\begin{aligned}
				\{x=(\xi,\  \eta) \mid x \in X,\  d(x,\  0)<1\} & =\left\{x=(\xi,\  \eta) \mid-\frac{1}{2} \leq \xi \leq \frac{1}{2},\  \eta=0\right\} \\
				& =X-\{(0,\ 1)\}
			\end{aligned}$$
			为  X  中开球,它是闭集但不是闭球:
			$$\{x=(\xi,\  \eta) \mid x \in X,\  d(x,\  0) \leq 1\}=X .$$
			从而距离空间中开球的闭包不等于闭球.
		\end{example}
	\end{note}
	\newpage
	\begin{problem}
		设$\alpha=(a_1,\ a_2,\ \cdots,\ a_n),\ (a_1\neq 0,\ n>1),\ A=\alpha^T\alpha,\ $求$A$的特征值和特征向量.
	\end{problem}
	\begin{solution}
		首先求$A=\alpha^T\alpha$的特征值\\
		法一:
		$$A^2=\alpha^T\alpha\alpha^T\alpha=\sum\limits_{i=1}^{n}a_i^2\alpha^T\alpha=\sum\limits_{i=1}^{n}a_i^2A.$$
		设$x$是$A$的属于特征值$\lambda$的特征向量,\ 则$Ax=\lambda x.$所以有
		$$A^2x=\sum\limits_{i=1}^{n}a_i^2 Ax=\sum\limits_{i=1}^{n}a_i^2\lambda x.$$
		而$A^2x=\lambda^2x,\ $
		所以$\lambda^2 x=\sum\limits_{i=1}^{n}a_i^2\lambda x\Rightarrow \left(\lambda^2-\lambda\sum\limits_{i=1}^{n}a_i^2\right)x=0,\ $\\
		由于$x\neq 0,\ $所以$\lambda^2-\lambda\sum\limits_{i=1}^{n}a_i^2=0\Rightarrow \lambda_1=\sum\limits_{i=1}^{n}a_i^2,\ \lambda_2=0.$\\
		由于
		$$A=\alpha^T\alpha=\left(\begin{matrix}
			a_1^2  & a_1a_2 & \cdots &a_1a_n\\
			a_2a_1 & a_2^2  & \cdots &a_2a_n\\
			\vdots & \vdots &        &\vdots\\
			a_na_1 & a_na_2 &\cdots  &a_n^2
		\end{matrix}\right)$$
		当$\lambda_2=0$时,\ 对$\lambda I-A$做初等行变换,\ 得
		$$\lambda_2I-A=\left(\begin{matrix}
			-a_1^2  & -a_1a_2 & \cdots &-a_1a_n\\
			-a_2a_1 & -a_2^2  & \cdots &-a_2a_n\\
			\vdots  & \vdots  &        &\vdots\\
			-a_na_1 & -a_na_2 &\cdots  &-a_n^2
		\end{matrix}\right)\xrightarrow[\text{乘}-a_i^{-1}]{\text{第$i$行}}\left(\begin{matrix}
		a_1    & a_2    & \cdots &a_n\\
		a_1    & a_2    & \cdots &a_n\\
		\vdots & \vdots &        &\vdots\\
		a_1    & a_2    &\cdots  &a_n
		\end{matrix}\right)\rightarrow\left(\begin{matrix}
		a_1    & a_2    & \cdots &a_n\\
		0      & 0      & \cdots &0  \\
		\vdots & \vdots &        &\vdots\\
		0    & 0      &\cdots  &0
		\end{matrix}\right)$$
	由于$a_1\neq 0,\ $所以$\mathrm{r}(\lambda_2I-A)=1,\ $故$(\lambda_2I-A)x=0$的基础解系中含有$n-1$个向量,\ 即对应$\lambda_2=0$的线性无关的特征向量个数为$n-1,\ $即$\lambda_2$的几何重数为$n-1,\ $故$\lambda_2$的代数重数$\geqslant 1.$又有$\lambda_1=\sum\limits_{i=1}^{n}a_i^2\neq 0,\ $所以$\lambda_2$的代数重数为$n-1.$\\
	法二:从$A^2=\sum\limits_{i=1}^{n}a_i^2A,\ $的两端同时乘$A^{n-2}$可以得到$A^n=\sum\limits_{i=1}^{n}a_i^2A^{n-1}.$\\
	同理有$A^nx=\sum\limits_{i=1}^{n}a_i^2A^{n-1}x\Rightarrow\lambda^{n-1}(\lambda-\sum\limits_{i=1}^{n}a_i^2)=0,\ $可以直接得到两个特征值的代数重数.\\
	法三:
	$$\begin{aligned}
		|\lambda I-A|&=\left|\begin{matrix}
			\lambda-a_1^2  & -a_1a_2 & \cdots &-a_1a_n\\
			-a_2a_1 & \lambda-a_2^2  & \cdots &-a_2a_n\\
			\vdots & \vdots &        &\vdots\\
			-a_na_1 & -a_na_2 &\cdots  &\lambda-a_n^2
		\end{matrix}\right|\xrightarrow[\text{加到第}i\text{行}(i=2,\ 3\cdots,\ n)]{\text{将第一行乘}\left(-\frac{a_i}{a_1}\right)}\left|\begin{matrix}
			\lambda - a_1a_1 & -a_1a_2 & \cdots & -a_1a_n\\
			-\frac{a_2}{a_1}\lambda & \lambda  & \cdots & 0\\
			\vdots         & \vdots &  & \vdots\\
			-\frac{a_n}{a_1}\lambda & 0 & \cdots & \lambda
		\end{matrix}\right|\\
	&\xrightarrow[\text{加到第一列$(j=2,\ 3,\ \cdots,\ n)$}]{\text{将第$j$列乘$\left(\frac{a_j}{a_1}\right)$}}\left|\begin{matrix}
		\lambda - \sum\limits_{i=1}^{n}a_i^2 & -a_1a_2 & \cdots & -a_1a_n\\
		0 & 0  & \cdots & 0\\
		\vdots         & \vdots &  & \vdots\\
		0 & 0 & \cdots & \lambda
	\end{matrix}\right|=\lambda^{n-1}\left(\lambda-\sum\limits_{i=1}^{n}a_i^2\right)
	\end{aligned}$$
	故$A$的特征值为$0(n-1\text{重}),\ \sum\limits_{i=1}^{n}a_i^2$
	下面求$A$的特征向量:\\
	当$\lambda_1=\sum\limits_{i=1}^{n}a_i^2$时,\ 由$(\lambda_1I-A)x=0$即
	$$\left(\begin{matrix}
		\sum\limits_{i=1}^{n}a_i^2-a_1^2  & -a_1a_2 & \cdots &-a_1a_n\\
		-a_2a_1 & \sum\limits_{i=1}^{n}a_i^2-a_2^2  & \cdots &-a_2a_n\\
		\vdots & \vdots &        &\vdots\\
		-a_na_1 & -a_na_2 &\cdots  &\sum\limits_{i=1}^{n}a_i^2-a_n^2
	\end{matrix}\right)\left(\begin{matrix}
	x_1\\
	x_2\\
	\vdots\\
	x_n
	\end{matrix}\right)=\left(\begin{matrix}
	0\\
	0\\
	\vdots\\
	0
	\end{matrix}\right)$$
	可得特征向量$x_1=(a_1,\ a_2,\ \cdots,\ a_n)^T$\\
	该特征向量还有第二种求法:
	$$A\alpha^T=(\alpha^T\alpha)\alpha^T=\sum\limits_{i=1}^{n}a_i^2\alpha^T.$$
	当$\lambda_2=0$时,\ 由$(\lambda_2I-A)x=0,\ $即
	$$\left(\begin{matrix}
		-a_1^2  & -a_1a_2 & \cdots &-a_1a_n\\
		-a_2a_1 & -a_2^2  & \cdots &-a_2a_n\\
		\vdots  & \vdots  &        &\vdots\\
		-a_na_1 & -a_na_2 &\cdots  &-a_n^2
	\end{matrix}\right)\left(\begin{matrix}
	x_1\\
	x_2\\
	\vdots\\
	x_n
	\end{matrix}\right)=\left(\begin{matrix}
	0\\
	0\\
	\vdots\\
	0
	\end{matrix}\right)$$
	可得$n-1$个线性无关的特征向量
	$$x_2=(a_2,\ -a_1,\ 0,\ \cdots,\ 0)^T,\ x_3=(a_3,\ 0,\ -a_1,\ \cdots,\ 0)^T,\ \cdots,\ x_n=(a_n,\ 0,\ 0,\ \cdots,\ -a_1)^T.$$
	取$P=(x_1,\ x_2,\ \cdots,\ x_n)$则有
	$$P^{-1}AP=\Lambda$$
%	=\left(\begin{matrix}
%		\sum\limits_{i=1}^{n}a_i^2 &   & &\\
%		& 0 & & \\
%		& & \ddots &\\
%		& & & 0
%	\end{matrix}\right)
	\end{solution}
	\newpage
	\begin{problem}
		证明线性回归的解析解为$w^{*}=(X^TX)^{-1}X^Ty$
	\end{problem}
	\begin{proof}
		\begin{enumerate}
			\item 先考虑单变量形式,\ 考虑对样本$\{(x_1,\ y_1),\ (x_2,\ y_2),\ \cdots,\ (x_n,\ y_n)\},\ $回归中考虑截距项,\ 将其写成矩阵形式:
			$$Y=\left(\begin{matrix}
						y_1      \\
						y_2     \\
						\vdots\\
						y_n    
					\end{matrix}\right);\qquad X=\left(\begin{matrix}
					1,\    &  x_1\\
					1,\    &  x_2\\
					\vdots& \vdots \\
					1   &   x_n
				\end{matrix}\right);\qquad B=\left(\begin{matrix}
				\omega     \\
				b     
			\end{matrix}\right)$$
		则回归方程可写成:$Y=XB+R,\ $其中$\omega,\ b$分别对应截距和斜率,\ 且
		$$R=\left(\begin{matrix}
			\varepsilon_1      \\
			\varepsilon_2     \\
			\vdots \\
			\varepsilon_n   
		\end{matrix}\right)$$
		为残差项,\ 目标是求得适当的$B$使得二范数$\left \|  R\right \|_2 $尽可能小
		$$L=(XB-Y)^T(XB-Y)=B^TX^TXB-2Y^TXB+Y^TY$$
		令$L$对$B$求导如下:
		$$\frac{\partial L}{\partial B}=2X^TXB-2X^TY$$
		令$\frac{\partial L}{\partial B}=0$可求得$L$的唯一极值为
		$$B^*=(X^TX)^{-1}X^TY$$
		\item 假设$n$个样本每个样本的观测方差之比为$\omega_{1}:\omega_{2}:\cdots:\omega_n,\ $不失一般性,\ 不妨令
		$$\sum_{i=1}^{n}\omega_i = 1$$
		记对角矩阵
		$$\Lambda = \left(\begin{matrix}
			\omega_{1} & 0 & \cdots & 0\\
			0 & \omega_2 & \cdots & 0\\
			\vdots & \vdots & \ddots & 0\\
			0 & 0 & \cdots & \omega_n
		\end{matrix}\right)$$
		那么加权线性回归里,\ 目标损失函数就变成了
		$$L=(XB-Y)^T\Lambda(XB-Y)=B^TX^TXB-2Y^T\Lambda XB+Y^T\Lambda Y$$
		同样,\ 令$L$对$B$求导有
		$$\frac{\partial L}{\partial B}=2X^T\Lambda XB-2X^T\Lambda Y$$
		令$\frac{\partial L}{\partial B}=0$可求得$L$的唯一极值为
		$$B^*=(X^TX)^{-1}X^T\Lambda Y$$
		\end{enumerate}
	\end{proof}
	\newpage
	\subsection{Three Theorems And Lemmas On Consistent Continuity And Continuity}
	$K$表示$\left[1,\ +\infty\right]$上非负一致连续函数类.
	\subsubsection{Lemma One}
	\begin{problem}
		若$f\in K,\ $则$\lim\limits_{x\rightarrow +\infty}\text{sup}\frac{f(x)}{x}<+\infty.$
	\end{problem}
	
	\begin{solution}
		设$\lim\limits_{x\rightarrow +\infty}\text{sup}\frac{f(x)}{x}=+\infty$,\ 则存在数列$\left\{x_k\right\},\ $使$x_k\rightarrow +\infty(k\rightarrow \infty),\ x_{k+1}-x_k>1,\ $且若$c_k=\frac{f\left(x_k\right)}{x_k},\ $则$c_{k+1}\ge c_k,\ c_k\rightarrow+\infty.$对每一$k,\ $有
		\begin{align*}
			|f(x_{k+1})-f(x_k)|&=c_{k+1}x_{k+1}-c_kx_k\\
			&\ge c_k\left(x_{k+1}-x_k\right).
		\end{align*}
		令$\varepsilon= 1,\ \delta$是满足$0<\delta < 1$的任意数,\ 则存在$k_0$使$k>k_0$时,\ $c_k>\frac{2}{\delta}.$取$k>k_0,\ $并选取$t_1,\ t_2,\ \dots,\ t_m,\ $使
		$$x_k=t_1<t_2<\dots<t_m=x_{k+1},\ $$
		这里$\frac{\delta}{2}<t_{i+1}-t_i<\delta(i=1,\ 2,\ \dots,\ m-1).$对某个$i$,\ 必有
		$$|f(t_{i+1})-f(t_i)|\ge c_k(t_{i+1}-t_i),\ $$
		因为否则有
		\begin{align*}
			|f(x_{k+1})-f(x_k)|&\le \sum\limits_{i=1}^{m-1}|f(t_{i+1})-f(t_i)|\\
			&<c_k\left(x_{k+1}-x_k\right),\ 
		\end{align*}
		与假设矛盾,\ 因此,\ 
		$$|f(t_{i+1})-f(t_i)|\ge c_k(t_{i+1}-t_i)>\frac{\delta}{2}\cdot\frac{2}{\delta}=1,\ $$
		于是$f$在$\left[1,\ +\infty\right]$上不一致连续,\ 矛盾. 
	\end{solution}
	
	$K^*$表示$\left[1,\ +\infty\right]$上的一切连续函数$f$,\ 使$f\in K,\ $且当$g\in K$时,\ $fg\in K.$
	\subsubsection{Lemma Two}
	\begin{problem}
		若$xf(x)\in K,\ $则任给$\varepsilon>0,\ $存在$\delta>0,\ $当$|x-y|<\delta$时,\ 就有
		$$x|f(x)-f(y)|<\varepsilon$$
	\end{problem}
	
	\begin{solution}
		因$xf(x)\in K,\ $据Lemma One,\ 存在常数$M>0,\ $使$f(x)<M.$此外,\ 存在$\delta>0,\ $使$\delta < \frac{\varepsilon}{2M},\ $且当$|x-y|<\delta$时,\ 有
		$$|xf(x)-yf(y)|<\frac{\varepsilon}{2}.$$
		因此
		\begin{align*}
			x|f(x)-f(y)|&=|xf(x)-xf(y)|=|xf(x)-yf(y)+yf(y)-xf(y)|\\
			&\le|xf(x)-yf(y)|+|f(y)||x-y|<\frac{\varepsilon}{2}+M\cdot\frac{\varepsilon}{2M}=\varepsilon.
		\end{align*} 
	\end{solution}
	
	\subsubsection{Theorem}
	\begin{problem}
		$f(x)\in K^*$的充要条件是$xf(x)\in K.$
	\end{problem}
	
	\begin{solution}
		必要性:因$x\in K,\ $故必要性成立.\\
		充分性:设$xf(x)\in K.$令$\varepsilon>0,\ g(x)\in K.$据Lemma One,\ 存在常数$M_1$及$M_2,\ $使
		$$f(x)<M_1,\ \qquad\frac{g(x)}{x}<M_2.$$
		据在$g(x)$上的假设及Lemma Two,\ 可选取$\delta > 0,\ $当$|x-y|<\delta$时
		$$|g(x)-g(y)|<\frac{\varepsilon}{2M_1},\ \qquad y|f(x)-f(y)|<\frac{\varepsilon}{2M_2}.$$
		因此,\ 当$|x-y|<\delta$时,\ 
		\begin{align*}
			|f(x)g(x)-f(y)g(y)|&\le f(x)|g(x)-g(y)|+\left(\frac{g(y)}{y}\right)y|f(x)-f(y)|\\
			&<M_1\cdot \frac{\varepsilon}{2M_1}+ M_2\cdot \frac{\varepsilon}{2M_2}\\
			&=\varepsilon.
		\end{align*}
		故$f(x)\in K^*.$ 
	\end{solution}